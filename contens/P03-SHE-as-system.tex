\section{Selective Harmonic Elimination as dynamical system}\label{Section3}

Inspired by the continuous nature of the optimization variable $u(\tau)$, we propose in this document the formulation from the optimal control. In this way we avoid the choice of the waveform, so that the optimization problem chooses the most convenient one in each case. So we look for $u (\tau)\in \mathcal{U}$, $\tau \in [0,\pi)]$, that has the desired Fourier coefficients.

We will use the fundamental theorem of differential calculus to rewrite the expression of the Fourier coefficients \eqref{an} as the evolution of a dynamical system. That is to say, for all $i$ and $j$
\begin{align*}
    \alpha_i(\tau) = \frac{2}{\pi}\int_0^\tau u(\tau) \cos(i\tau)d\tau 
    \Rightarrow
    \begin{cases} 
        \dot{\alpha_i}(\tau)  = \frac{2}{\pi}u(\tau)\cos(i\tau) \\  
        \alpha_i(0)  = 0       
    \end{cases}
	\\[10pt]
	\beta_j(\tau) = \frac{2}{\pi}\int_0^\tau u(\tau) \sin(j\tau)d\tau 
	\Rightarrow
	\begin{cases} 
		\dot{\beta}_j(\tau)  = \frac{2}{\pi}u(\tau)\sin(j\tau) \\  
		\beta_j(0) = 0       
	\end{cases}
\end{align*}

The evolution of the dynamical systems $\alpha_i(\tau)$ and $\beta_j(\tau) $ from the time $\tau = 0 $ to $\tau = \pi$ gives us the coefficients $a_i$ and $b_j$.

We introduce notation to refer to vectors $\bm{\alpha} = \{\alpha_i\}_{i\in\mathcal{E}_a}$ and $\bm{\beta} = \{\beta_j\}_{j\in\mathcal{E}_b}$.
%
In this way, the general SHE problem \eqref{SHEp} can be formulated as a control problem for a dynamical system where $\bm{\alpha}(\tau)$ and $\bm{\beta}(\tau)$ are the states and $u(\tau)$ is the control variable, and whose objective will be to bring the states from the origin to the objective vectors $\bm{a}_T$ and $\bm{b}_T $ in time $\tau = \pi$.

In order to obtain a compact expression of the problem that simplifies our understanding of it, we will introduce notation.
%
So if we consider a problem with sets of odd numbers:
\begin{align}
    \mathcal{E}_a = \{e_a^1,e_a^2,e_a^3,\dots,e_a^{N_a}\} \notag
    \\
    \mathcal{E}_b = \{e_b^1,e_b^2,e_b^3,\dots,e_b^{N_b}\}    
\end{align}
%
then we can define the vectors $\bm{\mathcal{D}}^\beta(\tau) \in \mathbb{R}^{N_a} $ y $ \bm{\mathcal{D}}^\beta(\tau) \in \mathbb{R}^{N_b} \ | \ \forall \tau \in (0,\pi]$ such that:
\begin{equation}
    \bm{\mathcal{D}}^\alpha(\tau) = 
        \begin{bmatrix} 
        \cos(e_a^1\tau) \\
        \cos(e_a^2\tau) \\
        \dots           \\
        \cos(e_a^{N_a}\tau) 
    \end{bmatrix},
    \bm{\mathcal{D}}^\beta(\tau) = 
    \begin{bmatrix} 
    \sin(e_b^1\tau) \\
    \sin(e_b^2\tau) \\
    \dots           \\
    \sin(e_b^{N_b}\tau) 
    \end{bmatrix} 
\end{equation}
%
So the dynamical system can be written as:
\begin{equation}
    \begin{aligned}
        \begin{cases}
            \dot{\bm{\alpha}}(\tau) = \big(\frac{2}{\pi}\big)\bm{\mathcal{D}}^\alpha(\tau) u(\tau) & \tau \in [0,\pi)\\
            \bm{\alpha}(0) = 0
        \end{cases} \\ \\
        \begin{cases}
            \dot{\bm{\beta}}(\tau)  = \big(\frac{2}{\pi}\big)\bm{\mathcal{D}}^\beta(\tau) u(\tau) & \tau \in [0,\pi) \\
            \bm{\beta}(0) = 0
        \end{cases}
    \end{aligned}
\end{equation}
Compressing the notation even more we can call the total state of the system $ \bm {x} (\tau) $ to the concatenation of the states $ \bm {\alpha} (\tau) $ and $ \bm {\beta} ( \tau) $ so that:
\begin{equation}
    \bm{x}(\tau) = \begin{bmatrix}
        \bm{\alpha}(\tau) \\  \bm{\beta}(\tau)
    \end{bmatrix} \hspace{1em}
    \bm{x}_0 = \begin{bmatrix}
        \bm{a}_T \\  \bm{b}_T
    \end{bmatrix} \hspace{1em}
    \bm{\mathcal{D}}(\tau) = \begin{bmatrix}
        \bm{\mathcal{D}}^\alpha(\tau) \\  
        \bm{\mathcal{D}}^\beta(\tau)
    \end{bmatrix}     
\end{equation}
%%
So for a pair of sets $ \mathcal {E} _a $ and $ \mathcal {E} _b $ we have the following associated dynamical system:
\begin{equation}
    \begin{cases}
        \dot{\bm{x}}(\tau) = \big(\frac{2}{\pi}\big)\bm{\mathcal{D}}(\tau) u(\tau)  & \tau \in [0,\pi)\\
        \bm{x}(0) = {0}
    \end{cases}
\end{equation}
Then we look for a function $ u (\tau) $ such that it leads the dynamical system to the point $ \bm {x} _T $, that is, the final state $ \bm {x} (\pi) $ is $ \bm {x}_0 $. Since in control theory one usually steers a dynamical susytem from a given initial datum to zero, we introduce the change of variables $\bm{x}(\tau)\mapsto \bm{x}(\tau) - \bm{x}_0$. Our dynamical system then becomes 
\begin{equation}\label{sys}
    \begin{cases}
        \dot{\bm{x}}(\tau) = -\big(\frac{2}{\pi}\big)\bm{\mathcal{D}}(\tau) u(\tau)  & \tau \in [0,\pi)\\
        \bm{x}(0) = \bm{x}_0
    \end{cases}
\end{equation}

In this way, if the state at time $\tau = \pi$ coincides with zero, then the Fourier coefficients of the control $u(\tau)$ which has steered it to that configuration are the ones associated with the final condition $\bm{x}_T$. 

\begin{remark}[Quarter-wave symmetry]
In the SHE literature, is usual to distinguish among the half-wave symmetry problem (addressed in the present paper) and the quarter-wave symmetry one. If we wanted to formulate the quarter-wave symmetry problem, we should simply modify system \ref{sys} in such a way that we eliminate the states $\alpha_j(\tau)$ and the integration is for $\tau \in (0,\pi/2]$.
\end{remark}