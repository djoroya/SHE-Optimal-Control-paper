\section{Conclusiones}\label{Section6}


We presented the SHE problem from the point of view of control theory. Nevertheless, comparing with methodologies where the commutation number is fixed a priori, our approximation is computationally more expensive. Notwithstanding that, the optimal control provides solutions in the entire range of the modulation index, although the number of solutions or their locations change dramatically.

This methodology for the SHE problem connects control theory with harmonic elimination. In this way, the SHE problem can be solved through classical tools.


\subsection{Quarter wave symetry}
We shall mention that, in the SHE literature \cite{Wu2009}, it is usual to distinguish among the half-wave symmetry problem (addressed in the present paper) and the quarter-wave symmetry one in which
    \begin{align*}
        u\left(\tau + \frac \pi2\right) = -u(\tau)\quad \mbox{for all}\; \tau \in \left[0,\frac \pi2\right).
    \end{align*}
    In quarter-wave symmetry, the SHE problem simplifies, as the Fourier coefficients $\{a_i\}_{i\in\mathcal E_a}$ turn out to be all zero. Hence, only the phases of the converter's signal can be controlled, while the half-wave SHE allows to deal with the amplitudes as well. It is worth to remark nonetheless that our optimal control formulation can be easily adapted to the quarter-wave symmetry problem by simply replacing the Fourier coefficients \eqref{eq:an} with
    \begin{align*}
        a_i = 0, \quad\quad b_j = \frac{4}{\pi} \int_0^{\frac \pi4} u(\tau)  \sin(j \tau)\,d\tau.
    \end{align*}
    Entonces podemos introducir el siguiente sistema dinámico:
    \begin{gather}
        \begin{cases}
            \displaystyle \dot{\bm{\beta}}(\tau) = \frac 2\pi \bm{\mathcal{D}}^\beta(\tau) u(\tau), & \tau \in [0,\pi/2)
            \\[6pt]
            \bm{\beta}(0) = \bm{b}_T
        \end{cases}\label{eq:CauchyReversed_4sym}
    \end{gather}
    Además del siguiente problema de control:
    \vspace{0.5em}
    \begin{problem}\label{pb:SHEpControl_4sym}
        Let $\mathcal{U}$ be defined as in \eqref{eq:Udef} and let $\mathcal{E} _b $ be a set of odd numbers with cardinality $ |\mathcal{E} _b| = N_b$. Given the vector $\bm{b}_T \in \mathbb{R}^{N_b} $. We look for $u:\in [0,\pi/2)\to\mathcal{U}$ such that the solution of \eqref{eq:CauchyReversed_4sym} with initial datum $\bm{x}(0)=\bm{x}_0$ satisfies $\bm{x}(\pi)=0$.
    \end{problem}
    Donde de la misma forma que en el problema con simetría de media onda la solución de este problema es también solución del problema SHE con simetría de cuarto de onda.

\subsection{Generalizations}


Consideramos que el tipo de penalizaciones utilizados en este problema puedes ser extendido a sistemas LTI:
\begin{gather}
	\min_{u \in \mathcal{U}}\;\frac 12 \|\bm{x}(T)\|^2 + \int_0^T \mathcal{L}(u(\tau))d\tau
	\\
    \notag \text{subject to: }\quad \begin{cases}
            \displaystyle \dot{\bm{x}}(\tau) = A(\tau)x(\tau) +B(\tau)u(\tau),  & \tau \in [0,T)\\[6pt]
            \bm{x}(0) = \bm{x}_0
    \end{cases}
\end{gather}

De manera que dado un término de penalización compatible con el Teorema \ref{th:GeneralP}, sea capaz de condicionar el control óptimo como un control digital.
