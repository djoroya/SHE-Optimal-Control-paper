
\subsection{Proof of Theorem \ref{th:SHEasDyn} (SHE as dinamical system)}

%As we anticipated in Section \ref{Section1}, the main contribution of the present paper is to provide a novel and alternative approach to the SHE problem, based on optimal control. As we shall see, this methodology will allow us avoiding the choice of the waveform, as the optimization process selects the most convenient one in each case. 

To this end, the starting point is to rewrite the expression of the Fourier coefficients \eqref{eq:an} as the evolution of a dynamical system. This can be easily done by means of the fundamental theorem of differential calculus as follows: for all $i,j\in\mathbb{N}$, let $\alpha_i$ and $\beta_j$ be the solutions of the following Cauchy problems
\begin{align}\label{eq:Cauchy}
	\begin{cases} 
		\displaystyle\dot{\alpha_i}(\tau)  = \frac{2}{\pi}u(\tau)\cos(i\tau), & \tau\in [0,\pi) 
		\\[6pt]  
		\alpha_i(0)  = 0       
	\end{cases} \notag 
	\\
	\\
	\begin{cases} 
		\displaystyle\dot{\beta}_j(\tau)  = \frac{2}{\pi}u(\tau)\sin(j\tau), & \tau\in [0,\pi) 
		\\[6pt]  
		\beta_j(0) = 0       
	\end{cases}\notag
\end{align}
Then 
\begin{align*}
	&\alpha_i(\tau)= \frac{2}{\pi}\int_0^\tau u(\zeta) \cos(i\zeta)\,d\zeta 
	\\[5pt]
	&\beta_j(\tau) = \frac{2}{\pi}\int_0^\tau u(\zeta) \sin(j\zeta)\,d\zeta 
\end{align*}
and the Fourier coefficients \eqref{eq:an} are given by $a_i=\alpha_i(\pi)$ and $b_j=\beta_j(\pi)$.  

Let now $\mathcal{E}_a$ and $\mathcal{E}_b$ be two sets of odd numbers, and denote
\begin{align*}
	\bm{\alpha} = \{\alpha_i\}_{i\in\mathcal{E}_a}, \quad \bm{\beta} = \{\beta_j\}_{j\in\mathcal{E}_b}.
\end{align*}
The dynamical systems \eqref{eq:Cauchy} can be rewritten in a vectorial form as:
\begin{align}\label{eq:CauchyVec}
	\begin{cases}
		\displaystyle \dot{\bm{\alpha}}(\tau) = \frac 2\pi \bm{\mathcal{D}}^\alpha(\tau) u(\tau), & \tau \in [0,\pi)
		\\[6pt]
		\bm{\alpha}(0) = 0
	\end{cases} \notag
	\\
	\\
	\begin{cases}
		\displaystyle\dot{\bm{\beta}}(\tau)  = \frac 2\pi \bm{\mathcal{D}}^\beta(\tau) u(\tau), & \tau \in [0,\pi) 
		\\[6pt]
		\bm{\beta}(0) = 0
	\end{cases}\notag 
\end{align}
where we use with $\bm{\mathcal{D}}^\beta(\tau) \in \mathbb{R}^{N_a} $ and $ \bm{\mathcal{D}}^\beta(\tau) \in \mathbb{R}^{N_b}$ define in \eqref{eq:DalphaDbeta}. 
Finally, compressing the notation even more, we can now denote 
\begin{align*}
	\bm{x}(\tau) = \begin{bmatrix} \bm{\alpha}(\tau) \\ \bm{\beta}(\tau) \end{bmatrix}, \quad
	\bm{\mathcal{D}}(\tau) = \begin{bmatrix} \bm{\mathcal{D}}^\alpha(\tau) \\ \bm{\mathcal{D}}^\beta(\tau) \end{bmatrix}     
\end{align*}
so that \eqref{eq:CauchyVec} becomes
\begin{align}\label{eq:CauchyCompact}
	\begin{cases}
		\displaystyle\dot{\bm{x}}(\tau) = \frac 2\pi\bm{\mathcal{D}}(\tau) u(\tau),  & \tau \in [0,\pi)
		\\[6pt]
		\bm{x}(0) = {0}
	\end{cases}
\end{align}
and the target coefficients of the SHE problem will be given by $\bm{x}_0:=[\bm{a}_T,\bm{b}_T]^\top=\bm{x}(\pi)$.

%Taking into account this new formulation, as we shall see in more detail in Section \ref{Section4}, Problem \ref{pb:SHEp} can now be recast as a control one for the dynamical systems \eqref{eq:CauchyCompact}, in which we look for a control function $u(\tau)$ steering the state $\bm{x}(\tau)$ from the origin to the target $\bm{x}_0:=[\bm{a}_T,\bm{b}_T]^\top$ in time $\tau = \pi$.

Moreover, since most often control problems are designed to drive the state of a given dynamical system to an equilibrium configuration, for instance the zero state, we introduce the change of variables $\bm{x}(\tau)\mapsto \bm{x}_0 - \bm{x}(\tau)$ which allows us to reverse the time in \eqref{eq:CauchyCompact}, thus obtaining the system show in \eqref{eq:CauchyReversed}. In this new configuration, the control function $u$ is now required to steer the solution of \eqref{eq:CauchyReversed} from the initial datum $\bm{x}_0$ to zero in time $\tau=\pi$. 