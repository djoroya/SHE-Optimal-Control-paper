
\begin{abstract}
    En este documento formularemos el problema de \emph{Selective Harmonic Elimination pulse-width modulation}(SHE-PWM) como el problema de control óptimo, con el fin de encontrar soluciones de ondas cuadradas sin prefijar el número de ángulos de conmunatación. Esta nueva perspectiva nos permite realizar un análisis sobre la continuidad de soluciones.
\end{abstract}
\tableofcontents

\section{Introducción} 

The \emph{Selective Harmonic Elimination} (SHE) problem is a modulation method that allows you to generate step signals\footnote{It should be noted that the step signals we are talking about are discontinuous functions defined in the interval $ [0,2 \pi] $ and that they can only take values in a small and finite set of values.}.
with a desired harmonic spectrum.
%
That is, given some Fourier coefficients, the SHE problem looks for the step waveform $ \{u (\tau) | \tau \in (0,2 \pi] \} $ whose Fourier coefficients are as required.
%
In general, half-wave symmetry, that is $ u (\tau + \pi) = - u (\tau) $, is required so in this work we will focus on this case.
%
In this context, if the waveform and number of switching are fixed, the problem SHE can be formulated as an optimization problem where the variables to be optimized are the switching locations.
%
Then we look for the switching locations that minimize the Euclidean distance between coefficients of the sought function and the given coefficients.
\newline 

%%%%%%%%%%%%%%%%%%%%%%%%%%%%
%
Although the problem SHE given a preset waveform is easily solvable, there are several difficulties in its application.
%
It is not possible to calculate the switching locations by means of a real-time optimization, which is why the solutions for different values of objective Fourier coefficients are pre-calculated.
%
When a non-precalculated solution is required, interpolations are carried out with the help of the other solutions to obtain it.
%
However, it is known that the space of solutions at the commutation angles is discontinuous with respect to a continuous variation of objective Fourier coefficients.
%
This is the reason why the interpolation of solutions is complex and sometimes impossible.
\newline
%%%%%%%%%%%%%%%%%%%%%%%%%%%%%%%%%%%%%%%%%%%%%%%%%%%%%%%%%%%

%
The nature of the appearance and fading of solutions through a continuous variation of the Fourier coefficients is unknown.
%
By calculating of the solutions it is known that the more switching locations are considered, the larger the continuous region of solutions.
%
This tells us that the number of switching locations required along a region of the solution space could change so that this formulation is not very flexible for a continuous description of the solutions.
\newline
%%%%%%%%%%%%%%%%%%%%%%%%%%%%%%%%%%%%%%%%%%%%%%%%%%%%%%%%%%%

%
In this document we will present the SHE problem as an optimal control problem, where the optimization variable is the signal $ u (\tau) $ defined in the entire interval $ [0,2 \pi) $.
%
Thus we will describe how in the problem of the Fourier coefficients of the function $ u (\tau) $ they can be seen as the final state of a system controlled by $ u (\tau) $.
%
So the optimization is performed among all the possible functions that satisfy $ | u (\tau) | <1 $ that can control the final state at the desired Fourier coefficients.
%
Then we will show how to design a control problem so that the solution is a step function.
%
Finally, we will show solutions to the SHE problem by formulating the optimal control, seeing how this methodology is versatile in the face of the variation in the number of commutations.%

