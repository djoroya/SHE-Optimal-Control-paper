\section{Introduction and motivations}\label{Section1}

Selective Harmonic Elimination (SHE) \cite{Rodriguez2002} is a well-known methodology in electrical engineering, employed to improve the performances of a converter by controlling the phase and amplitude of the harmonics in its output voltage. As a matter of fact, this technique allows to increase the power of the converter and, at the same time, to reduce its losses. 

In broad terms, the process consists in generating a \textit{control signal} with a desired harmonic spectrum, by modulating or eliminating some specific lower order frequencies. This signal is in the shape of a step function and is fully characterized by two features: 
\begin{itemize}
	\item[1.] the \textit{waveform}, i.e. the set of (constant) values the function may assume.
	\item[2.] the \textit{switching angles}, defining the points in the domain where the function changes from one constant value to another. 
\end{itemize}

Because of the growing complexity of modern electrical networks, consequence for instance of the high penetration of renewable energy sources, the demand in power of electronic converters is day by day increasing. For this and other reasons, SHE has been a preeminent research interest in the electrical engineering community, and a plethora of SHE-based techniques has been developed in recent years. An incomplete bibliography includes \cite{duranay2017selective,Janabi2020,Yang2017}.   

Nowadays, SHE is mostly based on offline computations to obtain the commutation patterns describing the control signal.  

\textcolor{red}{Add references and mention real-time approaches.}


The present document is organized as follows. In Section \ref{Section2}, we will present the mathematical formulation of the SHE problem and a classical resolution method based on an optimization process. In Section \ref{Section3}, we will formulate the SHE problem as a controlled dynamical system. In Section \ref{Section4}, we will present the optimal control problem allowing us to obtain the solutions to the SHE problem. In Section \ref{Section5}, we will present our numerical experiments. Finally, in Section \ref{Section6}, we will equation the conclusions of our study. 
