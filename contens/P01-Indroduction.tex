
\begin{abstract}
    En este documento formularemos el problema de \emph{Selective Harmonic Elimination pulse-width modulation}(SHE) como el problema de control óptimo, con el fin de encontrar soluciones de ondas cuadradas sin prefijar el número de ángulos de conmutación. 
    %
    Esta nueva perspectiva nos permite realizar un análisis sobre la continuidad de soluciones.
    %
    %In this document, we formulate the Selective Harmonic Elimination as a optimal control problem.
\end{abstract}
\tableofcontents

%%%%%%%%%%%%%%%%%%%%%%%%%%%%%%%%%%%%%%%%%%%%%%%%%%%%%%%%%%%%%%%%%%%%%%%%%%%%%%%%%%%%%%%%%%%%%%%%%%%%%%%%%%%%%%%%%%%%%%%%%%%%%%%%%%%%%%%%%%%%%%%%%%%%%%%%%%%%%%%%%%%%%%%%%%%%%%%%%%%%%%%%%%%%%%%%%%%%%%%%%%%%%%%%%%%%%%%%%%%%%%%%%%%%%%%%%%%%%%%%%%%%%%%%%%%%

\section{Introducción} 
 
%%%%%%%%%%%%%%%%%%%%%%%%%%%%%
%% DEFINICIÓN DEL PROBLEMA %%
%%%%%%%%%%%%%%%%%%%%%%%%%%%%% 

Selective Harmonic Elimination (SHE) \cite{Rodriguez2002} es una metodología de modulación de señal que permite generar señales escalón con el espectro armónico deseado. 
%
Es decir, nos permite generar una función $u(\tau)$ definida en $[0,2\pi)$ cuyos valores en todo el intervalo solo puedan estar contenida en un conjunto  finito de valores reales $\mathcal{U}$, y que además, tenga ciertos coeficientes de Fourier objetivo. 
%
Nos centraremos a funciones que cumplan simetría de media onda, es decir: $u(\tau + \pi) = -u(\tau) \ | \ \forall \tau \in [0,\pi)$. De esta manera la función $u(\tau)$ queda determinada por sus valores en el intervalo $[0,\pi)$. 
% 
%En la literatura, dependiendo de la cardinalidad del conjunto $\mathcal{U}$ el problema SHE recibe distintos nombres. 
%
%De esta manera, si consideramos el conjunto admisible como $\mathcal{U} = \{-1,1\}$, entonces el problema es llamado como SHE de dos niveles \cite{Konstantinou2010}, mientras que si el conjunto admisible es $\mathcal{U} = \{-1,0,1\}$ el problema es llamado SHE tri-nivel. 
%
\newline 

%%%%%%%%%%%%%%%%%%%%%%%%%
%% METODOLOGÍA CLÁSICA %%
%%%%%%%%%%%%%%%%%%%%%%%%%


Por lo general este problema se plantea como la búsqueda de una secuencia $S$ de números contenidos en $\mathcal{U}$ que nos defina qué valores tomará la función $u(\tau)$ y en que orden \cite{Konstantinou2010}.
%
De manera que dada la secuencia $S$ podemos centrarnos en buscar las localizaciones concretas donde se producen los cambios de valor de la función.
%
Por otra parte, las localizaciones de cambio dado la forma de onda $\mathcal{S}$ son llamados ángulos de conmutación \cite{Yang2015,Konstantinou2010,Sun1996}; de modo que procederemos a llamarlos de  la misma manera.
%
La búsqueda de las localizaciones dada la secuencia $S$ se puede abordar como un problema de minimización donde las variables son ángulos de conmutación mientras que el funcional de coste es la suma de la diferencia al cuadrado de los coeficientes de Fourier obtenidos y los deseados.
\newline 


%%%%%%%%%%%%%%%%%%%%%%%%%%%%%%%%%%%
%% PROBLEMAS METODOLOGÍA CLÁSICA %%
%%%%%%%%%%%%%%%%%%%%%%%%%%%%%%%%%%%
La búsqueda de la secuencia óptima no es evidente, este es un problema de combinatoria de los elementos de $\mathcal{U}$. 
%
Dado que \emph{a priori} no conocemos la cardinalidad de la secuencia $\mathcal{S}$, es decir el número de cambios necesarios,  el problema parece tener solo la solución de fijar el número de cambios de contar todas las posibles combinaciones para luego resolver un problema de optimización asociada a cada uno de ellas. 
%  
Teniendo en cuenta que el número de posibles secuencias depende de las cardinalidades del conjunto de admisible $\mathcal{U}$ y de la secuencia $S$, de la siguiente manera:  $|\mathcal{U}|^{|S|}$; es evidente que la complejidad de la búsqueda escala rápidamente con el tamaño de la secuencia.
% 
Este problema ha sido estudiado y abordado en \cite{Yang2015}, donde mediante tranformaciones algebráicas logran transformar el problema a un sistema de polinomos que contiene todas las posibles formas de ondas para un número fijo elementos de la secuencia $S$ y para un conjunto $\mathcal{U}$, sin embargo el número de posibles conmutaciones es prefijado. 
%
\newline

%%%%%%%%%%%%%%%%%%%%%%%%%%%%%%%%%%%
%% PROBLEMA DE LA INTERPOLACIÓN  %%
%%%%%%%%%%%%%%%%%%%%%%%%%%%%%%%%%%%
%
Por otra parte, es importante notar que la metodología SHE deberá responder a tiempo real para distintos coeficientes de Fourier objetivo con una latencia de $GHz$. 
%
Esto hace imposible la obtención de soluciones a tiempo real mediante optimización, por lo que es necesario el pre-calcular soluciones para que luego puedan ser interpoladas.
%
Sin embargo es bien conocido que fijado una secuencia $S$ la continuidad de las localizaciones de cambio con respecto a una variación continua de los coeficientes de Fourier objetivo es compleja. 
%
En la mayoría de los casos no se puede encontrar una solución continua para un intervalo largo, si no que es necesario cambiar la forma de onda $S$ según vamos tomando regiones de soluciones \cite{Yang2015,Yang2017}. 
% 
Esto hace complejo la interpolación de soluciones y la búsqueda de estas.
\newline
%%%%%%%%%%%%%%%%%%%%%%%%%%%%%%%%%%%%%%%%%%%%%%%%%%%%%%%%%%%%%%%%%%%%%%%%%%%%%%%%%%%%%%%%%%%%%%%%%%%%%%%%%%%%%%%%%%%%%%%%%%%%%%%%%%%%%%%%%%%%%%%%%%%%%%%%%%%%%%%%%%%%%%%%%%%%%%%%%%%%%%%%%%%%%%%%%%%%%%%%%%%%%%%%%%%%%%%%%%%%%%%%%%%%%%%%%%%%%%%%%%%%%%%%%%%%

%
In this document we will present the SHE problem as an optimal control problem, where the optimization variable is the signal $ u (\tau) $ defined in the entire interval $ [0,\pi) $.
%
Thus we will describe how in the problem of the Fourier coefficients of the function $ u (\tau) $ they can be seen as the final state of a system controlled by $ u (\tau) $.
%
So the optimization is performed among all the possible functions that satisfy $ | u (\tau) | <1 $ that can control the final state at the desired Fourier coefficients.
%
Then we will show how to design a control problem so that the solution is a step function.
%
\newline

Estructuraremos este documento de la siguiente manera: en la sección (1) Presentaremos la aproximación clasica de problema introduciendo conceptos básicos y el problema de optimización mostrando sus ventajas y desventajas; en la sección (2) Presentaremos el problema -she como un sistema dinámico de control; en la sección (3) Presentaremos el problema de control óptimo que nos permite obtener soluciones al problema; en la sección (4) Presentaremos experimentos numéricos para distintos niveles; y por último en la sección (5) mostraremos las conclusiones. 
