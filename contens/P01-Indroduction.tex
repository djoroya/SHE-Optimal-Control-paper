
\section{Introduction and motivations} 
 
Selective Harmonic Elimination (SHE) \cite{Rodriguez2002} is a well-known methodology allowing to generate signals in the form of a step function with a desired harmonic spectrum. 

In broad terms, this means to obtain a function $u(\tau)$ defined on $[0,2\pi)$ whose values in the whole interval can only belong to a finite set of real numbers $\mathcal{U}$ and which, in addition to that, has certain predetermined Fourier coefficients. 

Due to the application in power converters, we will focus on functions with half-wave symmetry, i. e. $u(\tau + \pi) = -u(\tau)$ for all $\tau \in [0,\pi)$. In this way, the function $u(\tau)$ is determined by its values in the interval $[0,\pi)$ and its Fourier series expansion takes the form
\begin{equation}
    u(\tau ) = \sum_{\underset{i\, odd}{i \in \mathbb{N}}} a_i \cos(i\tau)+ \sum_{\underset{j\, odd}{j \in \mathbb{N}}}  b_j \sin(j \tau), 
\end{equation}
where the coefficients $a_i$ and $b_j$ are given by
\begin{equation} \label{an}
    \begin{aligned}
    a_i = \frac{2}{\pi} \int_0^\pi u(\tau ) \cos(i \tau)\,d\tau, 
    \\
    b_j = \frac{2}{\pi} \int_0^\pi u(\tau)  \sin(j \tau)\,d\tau.
    \end{aligned}
\end{equation}
In view of this, the SHE problem can be formulated as:
\newline
\begin{problem}[SHE]\label{SHEp}
Let $\mathcal{U} = \{u_k\}_{k=1}^K$ be a finite subset of the interval $[-1,1]$ and let $\mathcal{E} _a $ and $\mathcal{E} _b $ be two sets of odd numbers with cardinalities $|\mathcal{E}_a| = N_a $ and $ |\mathcal{E} _b| = N_b$, respectively. Given the target vectors $\bm{a}_T \in \mathbb{R}^{N_a}$ and $\bm{b}_T \in \mathbb{R}^{N_b} $, we look for a function $u(\tau)$, $\tau \in [0,\pi)$, such that $u(\tau)$ can only take values within $\mathcal{U}$ and whose Fourier coefficients satisfy: 
\begin{gather}
	\notag a_i = (\bm{a}_T)_i, \quad\textrm{ for all } i \in \mathcal{E}_a,
	\\
	\notag b_j = (\bm {b}_T)_j, \quad\textrm{ for all } j \in \mathcal{E}_b.
\end{gather}
\end{problem}

Roughly speaking, this problem can be understood as finding a multi-set $\mathcal S = \{s_k\}_{k=1}^{\widehat K}$ with $s_k\in\mathcal U$, defining the values that the function $u(\tau)$ will assume and in which order they will appear (see \cite{Konstantinou2010}).
%
In this way, given the sequence $\mathcal S$ we can focus in finding the exact locations where the function $u(\tau)$ changes its values.
%
Following the terminology introduced in the SHE literature (\cite{Yang2015,Konstantinou2010,Sun1996}), we will refer to these locations of the changes in the values of the waveform $\mathcal{S}$ as \textit{switching angles}.
%
To finding of the switching angles given the sequence $S$ can be addressed as a minimization problem where the variables are the angles while the cost functional is the Euclidean distance between the obtained Fourier coefficients and the desired ones.

%%%%%%%%%%%%%%%%%%%%%%%%%%%%%%%%%%%
%% PROBLEMAS METODOLOGÍA CLÁSICA %%
%%%%%%%%%%%%%%%%%%%%%%%%%%%%%%%%%%%
Since the cardinality of $S$ is not known a priori, meaning that we do not know how many switches will be necessary, it appears that the only solution to the SHE problem consists in fix the number of changes and counting all the possible combinations, to later solve an optimization problem for each one of them.
%
Taking into account that the number of possible sequences $S$ is given by $|\mathcal{U}|^{|S|}$, it is evident that the complexity of the above approach increases rapidly with the cardinality of $S$.
%
This problem has been studied in \cite{Yang2015} where, through appropriate algebraic transformations, the authors are able to convert the SHE problem into a polynomial system whose solutions' set contains all the possible waveforms for a given set $\mathcal{U}$ and number of elements in the sequence $S$ which, however, is predetermined. 
%

On the other hand, we shall also mention that the SHE methodology has been developed to provide in real-time different target Fourier coefficients con with a $KHz$ latency. 
%
This makes impossible to find real-time solutions by optimization, making then necessary to pre-determine solutions that can later be interpolated.
%
Nevertheless, it is well-known that, fixed a sequence $S$, the continuity of the switching locations with respect to a continuous variation of the target Fourier coefficients may be quite cumbersome. 
%
In the majority of the cases, it is impossible to find a continuous solution in a large interval, an it is necessary to change the waveform $S$ while moving across different solution regions (\cite{Yang2015,Yang2017}). This makes difficult the interpolation of solutions and their finding.

In this document, we will present the SHE problem as an optimal control one, where the optimization variable is the signal $u(\tau)$ defined in the entire interval $[0,\pi)$. 
%
In particular, we will describe how the Fourier coefficients of the function $u(\tau)$ can be seen as the final state of a system controlled by $u (\tau)$. Hence, the optimization is performed among all the possible functions that satisfy $|u(\tau)|<1 $ and can control the final state at the desired Fourier coefficients. Then we will show how to design a control problem so that the solution is a step function.

The present document is organized as follows. In Section \ref{Section2}, we will present the classical formulations of the SHE problem os an optimization one, introducing some seminal concept and discussing its advantages and disadvantages. In Section \ref{Section3}, we will formulate the SHE problem as a controlled dynamical system. In Section \ref{Section4}, we will present the optimal control problem allowing us to obtain the solutions to the SHE problem. In Section \ref{Section5}, we will present our numerical experiments. Finally, in Section \ref{Section6}, we will equation the conclusions of our study. 
