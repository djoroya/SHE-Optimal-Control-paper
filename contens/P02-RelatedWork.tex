\section{Classical approach}\label{Section2}

The classical approximation uses the piece-wise definition of the function $u(\tau)$ and tries to exploit this information to simplify the problem.  In this definition, $u(\tau)$ can be fully characterized by the switching angles and the constant values it may assume. We define these two concepts as follows.

\begin{definition}[Switching angles]
Given a function $u:[0,\pi] \rightarrow \mathcal{U}$, the switching locations are the values of $\tau\in[0,\pi]$ where $u(\tau)$ changes its value discontinuously. We will denote the commutation angles as $\bm{\phi} = \{\phi_0,\phi_1,\dots,\phi_M,\phi_{M+1}\}$, where we have taken $\phi_0 = 0$ and $\phi_{M+1} = \pi$.
\end{definition}

\begin{definition}[Waveform]
Given $\mathcal U$ a finite subset of $[-1,1]$, we will call a waveform a finite set $S = \{s_1,s_2,s_3,\dots,s_{M+1}\}$ of elements of $\mathcal {U}$ with repetition.
\end{definition}

Then a waveform $S$ indicates the values that the function will take and in which order they will appear within the interval $[0,\pi) $, while $\bm{\phi}$ indicates the switching locations. Considering these two elements, we can rewrite the Fourier coefficients as
% \begin{equation}
%     a_i = \frac{2}{\pi} \int_0^\pi u(\tau) \cos(i \tau)d\tau  = \frac{2}{\pi} \sum_{k=1}^{M+1} 
%     %\int_{\phi_{k-1}}^{\phi_{k}} s_k \cos(i\tau)d\tau = 
%     \frac{2}{i\pi} \sum_{k=1}^{M+1} s_k \sin(i\tau)\,\Big|_{\phi_{k-1}}^{\phi_k} 
% \end{equation}
% \begin{equation}
%     b_j = \frac{2}{\pi} \int_0^\pi u(\tau ) \sin(j \tau)d\tau = 
%     %\frac{2}{\pi} \sum_{k =1}^{M+1} \int_{\phi_{k-1}}^{\phi_{k}} s_k \sin(j\tau)d\tau  
%     -\frac{2}{j\pi} \sum_{k=1}^{M+1} s_k \cos(j\tau)\, \Big|_{\phi_{k-1}}^{\phi_k} 
% \end{equation}
% Hence
\begin{equation}
    \begin{aligned}
        a_i(\bm{\phi}) & =  \frac{2}{i\pi} \sum_{k=1}^{M+1} s_k \Big[\sin(i\phi_k) -\sin(i\phi_{k-1})\Big]
        \\
        b_j(\bm{\phi}) & = \frac{2}{j\pi} \sum_{k=1}^{M+1} s_k \Big[\cos(j\phi_{k-1}) -\cos(j\phi_{k})\Big]
    \end{aligned}
\end{equation}
In this way, we can reformulate Problem \ref{SHEp} as follows.
\newline
\begin{problem}[Optimization for SHE]
Given a waveform $S$, we look for the switching angles $\bm{\phi}$ by means of the following minimization problem:
    \begin{gather}
        \min_{\bm{\phi} \in [0,\pi]^{M}} \sum_{i\in\mathcal{E}_a} \|a_T^i - a_i(\bm{\phi})\|^2 + \sum_{j\in \mathcal{E}_b} \|b_T^j - b_j(\bm{\phi})\|^2 
        \\
        \notag \text{subject to:}
        \\
        \notag 0 <\phi_1 < \phi_2 < \dots < \phi_{M-1} < \phi_{M} < \pi 
    \end{gather}
\end{problem}
In this formulation, the SHE problem converts in a minimization problem with restrictions which can be solved by well-known techniques. Since the problem has several minimizers, we shall solve it employing global optimizers. Furthermore, since the choice of the waveform is arbitrary, we shall proceed in the same way for each possible waveform. 