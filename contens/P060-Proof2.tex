
\subsection{Proof of Propsition \ref{prop:Hamiltoniano} (Hamiltonian Function)}\label{proof:Hamiltoniano}



In Problem \ref{pb:OCP_penalizado}, the penalization function $\mathcal L: \mathbb{R} \rightarrow \mathbb{R}$ will be chosen such that the optimal control $u^*$ only takes values in $\mathcal U$. Furthermore, the parameter $\epsilon$ should be small so that the solution minimizes the distance from the final state and the target.

Next we will study the optimality conditions of the problem, for a general $\mathcal L$, and then specify how $\mathcal L$ should be so that the optimal control $u^*$ only takes the allowed values in $\mathcal U$.

\subsubsection{Optimality conditions}

To write the optimality conditions of the problem we will use the Pontryagin minimum principle \cite[Chapter~2.7]{bryson1975applied}. With this purpose, it is necessary to first introduce the Hamiltonian function 
\begin{align*}\label{eq:hamil}
    H(u,\bm{p},\tau) = \epsilon \mathcal{L}(u) - \frac 2\pi\big(\bm{p}(\tau) \cdot \bm{\mathcal{D}}(\tau)\big)u(\tau),
\end{align*}
where $\bm{p}(\tau)$ is the so-called adjoint state, which is associated with the restriction imposed by the system. This vector has the same dimension of the state $\bm{x}$, so that
\begin{gather}
  \bm{x}(\tau) = \begin{bmatrix} \bm{\alpha}(\tau) \\ \bm{\beta}(\tau) \end{bmatrix} \Leftrightarrow 
  \bm{p}(\tau) = \begin{bmatrix} \bm{p}^\alpha(\tau) \\ \bm{p}^\beta(\tau) \end{bmatrix}.
\end{gather}
In what follows, we will enumerate the optimality conditions arising from the Pontryagin principle.
\begin{itemize}
    \item[1.] \textbf{Adjoint equation}: the ODE describing the evolution of the adjoint variable is given by 
    \begin{align*}
    	\dot{\bm{p}}(\tau) = -\nabla_x H(u(\tau),\bm{p}(\tau),\tau).
    \end{align*}
    In our case, since the Hamiltonian does not depend on the dynamics, we simply have
    \begin{align}\label{eq:equationP}
    	\dot{\bm{p}}(\tau) = 0,
    \end{align}
	that is, the adjoint state is constant in time.
	
	\item[2.] \textbf{Final condition of the adjoint}: As it is well-known, the adjoint equation is defined backward in time, meaning that its initial condition is actually a final one, posed at $\tau=\pi$. This final condition is given by 
    \begin{align*}
    	\bm{p}(\pi) = \nabla_{\bm{x}} \Psi(\bm{x}) = \bm{x}(\pi) - \bm{x}_T.
    \end{align*} 
	This, together with \eqref{eq:equationP}, tells us that
	\begin{align*}
		\bm{p}(\tau) = \bm{x}(\pi) - \bm{x}_T, \quad \mbox{ for all }\tau\in [0,\pi).
	\end{align*} 
    
    \item[3.] \textbf{Optimal  Waveform}: We known that 
    \begin{align*}
    	u^* = \argmin_{|u|<1} H(\tau,\bm{p}^*,u),
    \end{align*}
	so that, in this case, we can write
    \begin{gather}
        u^*(\tau) = \argmin_{|u|<1}  \left[\epsilon \mathcal{L}(u(\tau)) - \frac 2\pi \big(\bm{p}^* \cdot \bm{\mathcal{D}}(\tau)\big) u(\tau) \right].
    \end{gather}
    Therefore, this optimality condition reduces to the optimization of a function in a variable within the interval $ [- 1,1] $. 
\end{itemize}
\vspace{1em}
\begin{definition}
    Dado el problema \ref{pb:OCP_penalizado} definimos una función $\mathcal{H}_m:[-1,1]\rightarrow \mathbb{R}$ tal que:
    \begin{gather}\label{Hm}
        \mathcal{H}_m(u) = \epsilon \mathcal{L}(u) - mu  |  \forall m \in \mathbb{R}
    \end{gather}
\end{definition}
Es importante notar que la función $\mathcal{H}_m$ es el Hamiltoniano del sistema donde hemos remplazado el valor 
\begin{gather}
	[\bm{p}^* \cdot \bm{\mathcal{D}}(\tau)] = \sum_{i \in \mathcal{E}_a} p^*_\alpha \cos(i\tau) + \sum_{j \in \mathcal{E}_b} p^*_\beta \sin(j\tau) 
\end{gather}
por el parámetro $m$. De manera que el Hamiltoniano evaluado en la trayectoria óptima varía de manera continua en todo el intervalo $\tau \in [0,\pi)]$. 
%
Esta es la razón por la que el estudio de la función $\mathcal{H}_m$, una función uni-variable parametrizada por $m$ ,tiene implicaciones en el Problema \ref{pb:OCP_penalizado}.
\newline

\begin{definition}
    Dado el Problema \ref{pb:OCP_penalizado}  definimos una función $\mathcal{G}:\mathbb{R} \rightarrow [-1,1]$ tal que:
    \begin{gather}
        \mathcal{G}(m) = \argmin_{u \in [-1,1]} \mathcal{H}_m(u)
    \end{gather}
\end{definition}
\begin{definition}
    Dado el Problema \ref{pb:OCP_penalizado} definimos el conjunto $\mathcal{M}$ como:
    \begin{gather}
        \mathcal{M} = \{m \in \mathbb{R}\ | \ \mathcal{G}(m) \notin \mathcal{U} \}
    \end{gather}
\end{definition}



