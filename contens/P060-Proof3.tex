
\subsection{Proof of Theorem \ref{th:bang-bang}}\label{proof:bang-bang}

El caso binivel es el que tiene como conjunto admisible de controles $\mathcal{U}= \{-1,1\}$.
\vspace{1em}
Si $\mathcal{L}$ es concava en el intervalo entonces $\mathcal{H}_m$ también lo es. De manera que $G(m)$ solo puede tomar los valores $\{-1,1\}$.
