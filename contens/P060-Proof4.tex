
\subsection{Proof of Theorem \ref{th:PLP}}\label{proof:PLP}


To compute the minimum of $\mathcal{H}_m(u)$, we shall take into account that this function is not differentiable and the optimality condition then requires to work with the subdifferential $\partial\mathcal{L}(u)$, which given by
\begin{gather}
        \partial\mathcal{L}(u)= \begin{cases}
            \{u_1 + u_2  \}   & \text{if } \ u = u_1 \\
            %%%%%%%%%%%%%%%%%%%%%%%%%%%%%%%%%%%%%%%%%%%%%%%%
            \{u_k + u_{k+1}\}  & \text{if } \ u \in \ ]u_k,u_{k+1}[ \hspace{0.9em} \dagger\\
            %%%%%%%%%%%%%%%%%%%%%%%%%%%%%%%%%%%%%%%%%%%%%%%%
            [u_k+u_{k-1} ,  u_{k+1}+u_k] & \text{if} \ u = u_k \hspace{3.9em} \ddagger \\
            %%%%%%%%%%%%%%%%%%%%%%%%%%%%%%%%%%%%%%%%%%%%%%%%%
            \{u_{N_u} + u_{N_u-1}  \} & \text{if} \ u = u_{N_u} 
       \end{cases} \\
       \notag \dagger \ \forall k \in \{1,\dots,N_u-1\} \hspace{1em}
       \notag \ddagger  \ \forall k \in \{2,\dots,N_u-1\}
\end{gather} 

Hence, we have $\partial H_m = \epsilon\partial \mathcal{L} - m$. This means that, given $m\in \mathbb{R}$, we look for $u \in [-1,1]$ minimizing $\mathcal{H}_m(u)$. It is then necessary to determine whether zero belongs to $\partial \mathcal{H}_m(u)$.

\begin{itemize}
    \item \textbf{Case 1: $m \leq \epsilon(u_1+u_2)$}: since $m$ is less than the  minimum of all subdifferentials, then zero does not belong to any of the intervals we defined. Hence, the minimum is in one of the extrema
    \begin{gather}
        \argmin_{|u| < 1} \mathcal{H}_m(u) = u_1
    \end{gather} 
    \item \textbf{Case 2: $m = \epsilon(u_{k+1}+u_k) $}: taking into account that $\forall k \in \{1,\dots,N_u-1\}$,
    \begin{gather}
        \argmin_{|u| < 1} \mathcal{H}_m(u) = [u_k,u_{k+1}[ 
    \end{gather} 
    \item \textbf{Case 3: $\epsilon(u_k+u_{k-1})<m<\epsilon(u_{k+1}+u_k)$}: taking into account that $\forall k \in \{2,\dots,N_u-1\}$,
    \begin{gather}
        \argmin_{|u| < 1} \mathcal{H}_m(u) = u_k
    \end{gather}
    \item \textbf{Case 4: $m>\epsilon(u_{N_u}+u_{N_u-1})$}:
    \begin{gather}
        \argmin_{|u| < 1} \mathcal{H}_m(u) = u_{N_u}
    \end{gather} 
\end{itemize}

In other words, only when $m = \epsilon(u_{k+1}+u_k)$ the minima of the Hamiltonian belong to an interval. For all the other values of $m\in\mathbb{R}$, these minima are contained in $\mathcal{U}$. So that under a continuous variation of $m$, Case 2 can only occur pointwise. Recalling the optimal control problem $m(\tau) = [\bm{p}(\tau) \cdot \bm{\mathcal{D}}(\tau)]$, we can notice that Case 2 corresponds to the instants $\tau$ of change of value.



