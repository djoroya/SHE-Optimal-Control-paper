
\section{Classical approach to the SHE problem}\label{sec:ClassicalSHEproblem}

As we anticipated in Section \ref{Section1}, the control signal $u$ is fully characterized by its waveform and the switching angles, to which we now give a precise definition. 
\vspace{1em}
\begin{definition}[Wave-form]\label{def:waveform}
Given the finite set $\mathcal U$ defined in \eqref{eq:Udef} and $M\in\mathbb{N}$, we will call a waveform any possible $(M+1)$-tuple $\mathcal S = (s_m)_{m=1}^{M+1}$ with $s_m\in \mathcal U$ for all $m=1,\ldots,M+1$.
\end{definition}
\vspace{1em}
\begin{definition}[Switching angles]\label{def:switchingAngles}
Given the finite set $\mathcal U$ defined in \eqref{eq:Udef}, $M\in\mathbb{N}$ and a piece-wise constant function $u:[0,\pi) \rightarrow \mathcal{U}$, we shall refer as switching angles $\bm{\phi} = \{\phi_m\}_{m=0}^{M+1}\subset[0,\pi]$, with $\phi_0 = 0$ and $\phi_{M+1} = \pi$, to the points in the domain $[0,\pi)$ where $u$ changes its value. 
\end{definition} 

The idea of classical aproach of SHE problem if exploit these feature to reduce the complexity of the problem. In view of the above definitions, we can provide the following explicit expression for $u$:
\begin{align}\label{eq:uExpl}
	&u = \sum_{m=1}^{M+1} s_m\chi_{[\phi_m,\phi_{m+1}]}
	\\
	&s_m\in\mathcal S, \;\phi_m\in{\bm{\phi}}, \quad \mbox{for all } m=1,\ldots,M+1, \notag 
\end{align} 
where we denoted by $\chi_{[\phi_m,\phi_{m+1}]}$ the characteristic function of the interval $[\phi_m,\phi_{m+1}]$.

Besides, taking into account \eqref{eq:uExpl}, a direct computation yields that the Fourier coefficients \eqref{eq:an} are given by
\begin{align*}
	& a_i = a_i(\bm{\phi}) =  \frac{2}{i\pi} \sum_{k=1}^{M+1} s_k \Big[\sin(i\phi_k) -\sin(i\phi_{k-1})\Big]
	\\
	& b_j = b_j(\bm{\phi}) = \frac{2}{j\pi} \sum_{k=1}^{M+1} s_k \Big[\cos(j\phi_{k-1}) -\cos(j\phi_{k})\Big]
\end{align*}
Given a waveform $\mathcal S$, Problem \ref{pb:SHEp} then reduces to find the switching locations $\bm{\phi}$ (see \cite{Yang2015,Konstantinou2010,Sun1996}). This can be cast as a minimization problem in the variables $\{\phi_m\}_{m=0}^{M+1}$, where the cost functional is the Euclidean distance between the obtained Fourier coefficients $\{a_i(\bm{\phi}),b_j(\bm{\phi})\}$ and the targets $(\bm{a},\bm{b})\in \mathbb{R}^{N_a}\times \mathbb{R}^{N_b}$.
\newline
\begin{problem}[Optimization for SHE]
Given a waveform $\mathcal S$ and a step function $u$ in the form \eqref{eq:uExpl}, we look for the switching angles locations $\bm{\phi}$ by means of the following minimization problem:
\begin{align}
	&\min_{\bm{\phi} \in [0,\pi]^{M+1}} \left(\sum_{i\in\mathcal{E}_a} \|a_T^i - a_i(\bm{\phi})\|^2 + \sum_{j\in \mathcal{E}_b} \|b_T^j - b_j(\bm{\phi})\|^2\right)\notag 
	\\[10pt]
	&\mbox{subject to: } 0 = \phi_0 <\phi_1 < \ldots < \phi_{M} < \phi_{M+1} = \pi \notag 
\end{align}
\end{problem}
Next we will mention some problems of this formulation:
\begin{enumerate}
    \item \textbf{Combinatory problem}: Since the cardinality of $\mathcal S$ is not known a priori, meaning that we do not know how many switches will be necessary to reach the desired values of the Fourier coefficients, a common approach to solve the SHE problem consists in fixing the number of changes and generating all the possible combinations of elements of $\mathcal S$, to later solve an optimization problem for each one of them. Nevertheless, taking into account that the number of possible tuples $\mathcal S$ is given by $|\mathcal{U}|^{|\mathcal S|}$, it is evident that the complexity of the above approach increases rapidly. This problem has been studied in \cite{Yang2015} where, through appropriate algebraic transformations, the authors are able to convert the SHE problem into a polynomial system whose solutions' set contains all the possible waveforms for a given set $\mathcal{U}$ and number of elements in the sequence $\mathcal S$.  However, the number of posibles switches is pre-fixed. In our case, we extend this idea and signal is totally free.

	\item \textbf{Continuity problem}: It is well-known that, fixed a waveform $S$, the continuity of the switching locations with respect to a continuous variation of the target Fourier coefficients may be quite cumbersome \cite{Yang2017}. 
	%
	This feature makes the search very dificult, because if you take a some inverval where you can solve the SHE problem, you need to assure you that the solution exist in this interval.
	%
	In this case, the SHE as control problem can change the waveform under a small variation of the target Fourier coefficients.
\end{enumerate}
%
%

%
%
