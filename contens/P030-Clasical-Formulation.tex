
\section{Classical approach to the SHE problem}

En esta sección presentaremos la aproximación actual que se utiliza para abordar el problema, y mencionaremos sus problemas asociado. As we anticipated in Section \ref{Section1}, the control signal $u$ is fully characterized by its waveform and the switching angles, to which we now give a precise definition.
\newline
\begin{definition}[Wave-form]\label{def:waveform}
Given the finite set $\mathcal U$ defined in \eqref{eq:Udef} and $M\in\mathbb{N}$, we will call a waveform any possible $(M+1)$-tuple $\mathcal S = (s_m)_{m=0}^{M+1}$ with $s_m\in \mathcal U$ for all $m=0,\ldots,M+1$.
\end{definition}
\vspace{1em}
\begin{definition}[Switching angles]\label{def:switchingAngles}
Given the finite set $\mathcal U$ defined in \eqref{eq:Udef}, $M\in\mathbb{N}$ and a piece-wise constant function $u:[0,\pi) \rightarrow \mathcal{U}$, we shall refer as switching angles $\bm{\phi} = \{\phi_m\}_{m=0}^{M+1}\subset[0,\pi]$, with $\phi_0 = 0$ and $\phi_{M+1} = \pi$, to the points in the domain $[0,\pi)$ where $u$ changes its value. 
\end{definition}

In view of the above definitions, we can provide the following explicit expression for $u$:
\begin{align}\label{eq:uExpl}
	&u = \sum_{m=0}^{M+1} s_m\chi_{[\phi_m,\phi_{m+1}]}
	\\
	&s_m\in\mathcal S, \;\phi_m\in{\bm{\phi}}, \quad \mbox{for all } m=0,\ldots,M+1, \notag 
\end{align}
where we denoted by $\chi_{[\phi_m,\phi_{m+1}]}$ the characteristic function of the interval $[\phi_m,\phi_{m+1}]$.

Besides, taking into account \eqref{eq:uExpl}, a direct computation yields that the Fourier coefficients \eqref{eq:an} are given by
\begin{align*}
	& a_i = a_i(\bm{\phi}) =  \frac{2}{i\pi} \sum_{k=1}^{M+1} s_k \Big[\sin(i\phi_k) -\sin(i\phi_{k-1})\Big]
	\\
	& b_j = b_j(\bm{\phi}) = \frac{2}{j\pi} \sum_{k=1}^{M+1} s_k \Big[\cos(j\phi_{k-1}) -\cos(j\phi_{k})\Big]
\end{align*}
Given a waveform $\mathcal S$, Problem \ref{pb:SHEp} then reduces to find the switching locations $\bm{\phi}$ (see \cite{Yang2015,Konstantinou2010,Sun1996}). This can be cast as a minimization problem in the variables $\{\phi_m\}_{m=0}^{M+1}$, where the cost functional is the Euclidean distance between the obtained Fourier coefficients $\{a_i(\bm{\phi}),b_j(\bm{\phi})\}$ and the targets $(\bm{a},\bm{b})\in \mathbb{R}^{N_a}\times \mathbb{R}^{N_b}$.
\newline
\begin{problem}[Optimization for SHE]
Given a waveform $\mathcal S$ and a step function $u$ in the form \eqref{eq:uExpl}, we look for the switching angles $\bm{\phi}$ by means of the following minimization problem:
\begin{align}
	&\min_{\bm{\phi} \in [0,\pi]^{M}} \left(\sum_{i\in\mathcal{E}_a} \|a_T^i - a_i(\bm{\phi})\|^2 + \sum_{j\in \mathcal{E}_b} \|b_T^j - b_j(\bm{\phi})\|^2\right)\notag 
	\\[10pt]
	&\mbox{subject to: } 0 = \phi_0 <\phi_1 < \ldots < \phi_{M} < \phi_{M+1} = \pi \notag 
\end{align}
\end{problem}
A continuación mecionaremos algunos problemas de esta formulación:
\begin{enumerate}
    \item \textbf{Combinatory problem}: Since the cardinality of $\mathcal S$ is not known a priori, meaning that we do not know how many switches will be necessary to reach the desired values of the Fourier coefficients, a common approach to solve the SHE problem consists in fixing the number of changes and generating all the possible combinations of elements of $\mathcal S$, to later solve an optimization problem for each one of them. Nevertheless, taking into account that the number of possible tuples $\mathcal S$ is given by $|\mathcal{U}|^{|\mathcal S|}$, it is evident that the complexity of the above approach increases rapidly. This problem has been studied in \cite{Yang2015} where, through appropriate algebraic transformations, the authors are able to convert the SHE problem into a polynomial system whose solutions' set contains all the possible waveforms for a given set $\mathcal{U}$ and number of elements in the sequence $\mathcal S$ which, however, is predetermined. 
	\item \textbf{Real-time problem}: On the other hand, we shall also mention that the SHE methodology has been developed to provide in real-time different target Fourier coefficients con with a $KHz$ latency. This makes impossible to find real-time solutions by optimization, making then necessary to pre-determine solutions that can later be interpolated.
	\item \textbf{Continuity problem}: Nevertheless, it is well-known that, fixed a sequence $S$, the continuity of the switching locations with respect to a continuous variation of the target Fourier coefficients may be quite cumbersome. 
\end{enumerate}
%
%

%
%
