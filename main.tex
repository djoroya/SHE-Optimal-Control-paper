
%\documentclass{elsart}               % The use of LaTeX2e is preferred.

\documentclass[twocolumn]{autart}    % Enable this line and disable the 
                                     % preceding line to obtain a two-column 
                                     % document whose style resembles the
                                     % printed Automatica style.


\usepackage{graphicx}          % Include this line if your 
                               % document contains figures,
%\usepackage[dvips]{epsfig}    % or this line, depending on which
                               % you prefer.

\usepackage{graphicx}

\usepackage{bm}

\usepackage{amsmath}
\usepackage{amssymb}
\usepackage{ntheorem}

\usepackage{hyperref}

\newtheorem{problem}{Problem}[section]
\newtheorem{definition}{Definition}[section]
\newtheorem{remark}{Remark}[section]

\numberwithin{equation}{section}

%
\DeclareMathOperator*{\argmin}{arg\,min}

\usepackage{xcolor}

\begin{document}

\begin{frontmatter}
\runtitle{Selective Harmonic Elimination via Optimal Control}   % Running title for regular 
                                              % papers but only if the title  
                                              % is over 5 words. Running title 
                                              % is not shown in output.

\title{Selective Harmonic Elimination via Optimal Control} % Title, preferably not more 
                                                 % than 10 words.

\thanks[footnoteinfo]{This paper was not presented at any IFAC 
meeting.}

\author[FD,UD]{Umberto Biccari}\ead{umberto.biccari@deusto.es},    % Add the 
\author[UAM,FD]{Carlos Esteve}\ead{carlos.esteve@uam.es},               % e-mail address 
\author[UD]{Jes\'us Oroya}\ead{djoroya@deusto.es}  % (ead) as shown
\address[FD]{Chair of Computational Mathematics, Fundaci\'on Deusto, Avenida de las Universidades 24, 48007 Bilbao, Basque Country, Spain.}  %
\address[UD]{Universidad de Deusto, Avenida de las Universidades 24, 48007 Bilbao, Basque Country, Spain.}  %
\address[UAM]{Departamento de Matem\'aticas, Universidad Aut\'onoma de Madrid, 28049 Madrid, Spain.}  % Please supply                                              

          
\begin{keyword}                           % Five to ten keywords,  
Selective Harmonic Elimination; Finite Set Control, Piecewise Linear function.               % chosen from the IFAC 
\end{keyword}                             % keyword list or with the 
                                          % help of the Automatica 
                                          % keyword wizard


\begin{abstract}                          % Abstract of not more than 200 words.
El problema de \emph{Selective Harmonic Elimination pulse-width modulation}(SHE) es planteado como el problema de control óptimo, con el fin de encontrar soluciones de ondas escalón sin prefijar el número de ángulos de conmutación. De esta manera, la metodología de control óptimo es capaz de encontar la forma de onda óptima y de encontrar la localizaciones de los ángulos de conmutación, incluso sin prefijar el número de conmutaciones. Este es un nuevo enfoque para el problema SHE en concreto y para los sistemas de control con un conjunto finito de controles admisibles en general.

\end{abstract}

\end{frontmatter}


\section{Introduction and motivations} 

Selective Harmonic Elimination (SHE) \cite{Rodriguez2002} is a well-known methodology in electrical engineering, employed to increase the performances of a converter by controlling the phase and amplitude of the harmonics in its output voltage. In broad terms, the process consists in generating a signal in the form of a step function with a desired harmonic spectrum, by modulating or eliminating certain relevant frequencies. This allows to increase the power of the converter and, at the same time, to reduce its losses.  

Because of the growing complexity of modern electrical networks, consequence for instance of the high penetration of renewable energy sources, the demand in power of electronic converters is day by day increasing. For this and other reasons, SHE has been a preeminent research interest in the electrical engineering community, and a plethora of SHE-based techniques has been developed in recent years. An incomplete bibliography includes \cite{duranay2017selective,Janabi2020,Yang2017}.   

Nowadays, SHE is mostly based on offline computations to obtain the commutation patterns describing what is called the \textit{waveform} of the converter's signal, that is, ...




The present document is organized as follows. In Section \ref{Section2}, we will present the mathematical formulation of the SHE problem and a classical resolution method based on an optimization process. In Section \ref{Section3}, we will formulate the SHE problem as a controlled dynamical system. In Section \ref{Section4}, we will present the optimal control problem allowing us to obtain the solutions to the SHE problem. In Section \ref{Section5}, we will present our numerical experiments. Finally, in Section \ref{Section6}, we will equation the conclusions of our study. 

\section{Formulaciones clásicas}

A continuación mencionaremos trabajos relacionados con el problema SHE mencionando su fortalezas y limitaciones con el fin de enfatisar las ventajas de considerar el problema SHE como un problema de control. Entonces como hemos mencionado el problema SHE es típicamente formulado como problema de óptimización sobre las localizaciones de los ángulos de conmutación, sin embargo en el caso en el que la función escalón pueda tomar más de dos niveles no solo deberemos indicar en que lugar se realiza la conmutación sino también qué forma de onda estamos siguiendo. Es por ello que las formulaciones de convertidores de dos y tres niveles están claramente diferenciadas por lo que continuaremos su descripción en secciones separadas.
\newline 

El problema \emph{Selective Harmonic Elimination}(SHE) consiste en la búsqueda de una función $f(\tau )$ definida en el intervalo $[0,2\pi]$, fijados unos pocos coeficientes de Fourier. Esta función $f(\tau)$ solo podrá tomar dos posibles valores $\{-1,1\}$.
%
Nos centraremos en concreto en las funciones $f(\tau)$ con simetría de media onda, es decir funciones tal que $f(\tau) = -f(\tau + \pi)$, por lo que la descripción de la función $f(\tau)$ queda determinada con sus valores en el intervalo $\tau \in [0,\pi]$. De esta forma, nos referiremos a una función $\{ f(\tau)  | \tau \in [0,\pi] \}$ cuyo desarrollo en serie de Fourier se puede escribir como:

\begin{gather}
    f(\tau ) = \sum_{i \in odd} a_i \cos(i\tau)+ \sum_{j \in odd}  b_j \sin(j \tau) 
\end{gather}

Donde $a_i$ y $b_j$  son:
\begin{gather}
    a_i = \frac{2}{\pi} \int_0^\pi f(\tau ) \cos(i \tau)d\tau \ | \ \forall i \ odd \label{an}\\
    b_j = \frac{2}{\pi} \int_0^\pi f(\tau)  \sin(j \tau) d\tau \ | \ \forall j \ odd \label{bn}
\end{gather}

A continuación introduciremos los problemas SHE para dos niveles, tres niveles y multi-nivel explicando la metodología que se suele aplicar para la solución de estos casos.


\subsection{Formulación clásica para dos y tres niveles}
Entonces podemos formular el problema SHE para dos niveles de la siguiente manera:

\begin{problem}[SHE para dos niveles]\label{SHEp}
    Dado dos conjuntos de números impares $\mathcal{E}_a$ y $\mathcal{E}_b$ con cardinalidades $|\mathcal{E}_a| = N_a$ y  $|\mathcal{E}_b| = N_b$ respectivamente, y dado los vectores objetivo $\bm{a}_T  \in \mathbb{R}^{N_a}$ y $\bm{b}_T  \in \mathbb{R}^{N_b}$, buscamos una función  $\{f(\tau ) \ | \ \tau \in (0,\pi)\}$ tal que $f(\tau)$ solo pueda tomar los valores  $\{-1,1\}$ y cuyos coeficientes de Fourier satisfagan: $ a_i = (\bm{a}_T)_i \ | \ \forall i \in \mathcal{E}_a$ y  $b_j = (\bm{b}_T)_j \ \forall \ | \  j \in \mathcal{E}_b$. 
\end{problem}


Además en el caso en el que  consideremos que  $\{f(\tau) | \tau \in [0,\pi]\}$ puede tomar valores en el conjunto conjunto $\{1,0\}$ obtenemos tres niveles en la función definida en el intervalo total $[0,2\pi]$. Esto se debe a la simetría de media onda. De esta manera podemos formular el problema de tres niveles de manera equivalente como:

\begin{problem}[SHE para tres niveles]\label{SHEp}
    Dado dos conjuntos de números impares $\mathcal{E}_a$ y $\mathcal{E}_b$ con cardinalidades $|\mathcal{E}_a| = N_a$ y  $|\mathcal{E}_b| = N_b$ respectivamente, 
    %
    y dado los vectores objetivo $\bm{a}_T  \in \mathbb{R}^{N_a}$ y $\bm{b}_T  \in \mathbb{R}^{N_b}$, 
    %
    buscamos una función  $\{f(\tau ) \ | \ \tau \in (0,\pi)\}$ tal que $f(\tau)$ solo pueda tomar los valores  $\{0,1\}$ y cuyos coeficientes de Fourier satisfagan: $ a_i = (\bm{a}_T)_i \ | \ \forall i \in \mathcal{E}_a$ y  $b_j = (\bm{b}_T)_j \ \forall \ | \  j \in \mathcal{E}_b$. 
\end{problem}


\subsection{Formulación clásica para multi-nivel}

Por último, existe la posibilidad de considerar un conjunto  discreto como posibles valores que pueda tomar la función $f(\tau)$ de esta manera si llamamos $\Omega_f$ al conjunto de valores discretos que puede tomar $f(\tau)$ podemos definir el problema de SHE multi-nivel de la siguiente manera.

\begin{problem}[SHE  multi-nivel]\label{SHEp_2LVL}
    Dado dos conjuntos de números impares $\mathcal{E}_a$ y $\mathcal{E}_b$ con cardinalidades $|\mathcal{E}_a| = N_a$ y  $|\mathcal{E}_b| = N_b$ respectivamente, y dado los vectores objetivo $\bm{a}_T  \in \mathbb{R}^{N_a}$ y $\bm{b}_T  \in \mathbb{R}^{N_b}$, buscamos una función  $\{f(\tau ) \ | \ \tau \in (0,\pi)\}$ tal que $f(\tau)$ solo pueda tomar los valores en una discretización $\Omega_f$ del intervalo $[-1,1]$ y cuyos coeficientes de Fourier satisfagan: $ a_i = (\bm{a}_T)_i \ | \ \forall i \in \mathcal{E}_a$ y  $b_j = (\bm{b}_T)_j \ \forall \ | \  j \in \mathcal{E}_b$. 
\end{problem}





\section{SHE as a dynamical system}\label{Section3}

As we anticipated in Section \ref{Section1}, the main contribution of the present paper is to provide a novel and alternative approach to the SHE problem, based on the optimal control. As we shall see, this methodology will allow us avoiding the choice of the waveform, as the optimization process chooses the most convenient one in each case. 

To this end, the starting point is to rewrite the expression of the Fourier coefficients \eqref{eq:an} as the evolution of a dynamical system. This can be easily done by means of the fundamental theorem of differential calculus as follows: for all $i,j\in\mathbb{N}$, let $\alpha_i$ and $\beta_j$ be the solutions of the following Cauchy problems
\begin{align}\label{eq:Cauchy}
	\begin{cases} 
		\displaystyle\dot{\alpha_i}(\tau)  = \frac{2}{\pi}u(\tau)\cos(i\tau), & \tau\in [0,\pi) 
		\\[6pt]  
		\alpha_i(0)  = 0       
	\end{cases} \notag 
	\\
	\\
	\begin{cases} 
		\displaystyle\dot{\beta}_j(\tau)  = \frac{2}{\pi}u(\tau)\sin(j\tau), & \tau\in [0,\pi) 
		\\[6pt]  
		\beta_j(0) = 0       
	\end{cases}\notag
\end{align}
Then 
\begin{align*}
	&\alpha_i(\tau):= \frac{2}{\pi}\int_0^\tau u(\zeta) \cos(i\zeta)\,d\zeta 
	\\[5pt]
	&\beta_j(\tau) = \frac{2}{\pi}\int_0^\tau u(\zeta) \sin(j\zeta)\,d\zeta 
\end{align*}
and the Fourier coefficients \eqref{eq:an} are given by $a_i=\alpha_i(\pi)$ and $b_j=\beta_j(\pi)$.  

Let now
\begin{align*}
	\mathcal{E}_a = \{e_a^1,e_a^2,e_a^3,\dots,e_a^{N_a}\}, \quad \mathcal{E}_b = \{e_b^1,e_b^2,e_b^3,\dots,e_b^{N_b}\}    
\end{align*}
be two sets of odd numbers, and denote
\begin{align*}
	\bm{\alpha} = \{\alpha_i\}_{i\in\mathcal{E}_a}, \quad \bm{\beta} = \{\beta_j\}_{j\in\mathcal{E}_b}.
\end{align*}
Then, for any $\tau\in [0,\pi)$, we can define the vectors $\bm{\mathcal{D}}^\beta(\tau) \in \mathbb{R}^{N_a} $ and $ \bm{\mathcal{D}}^\beta(\tau) \in \mathbb{R}^{N_b}$ as:
\begin{align*}
	\bm{\mathcal{D}}^\alpha(\tau) = 
	\begin{bmatrix} 
		\cos(e_a^1\tau) \\ \cos(e_a^2\tau) \\ \vdots \\ \cos(e_a^{N_a}\tau) 
	\end{bmatrix},
	\quad \bm{\mathcal{D}}^\beta(\tau) = 
	\begin{bmatrix} 
		\sin(e_b^1\tau) \\ \sin(e_b^2\tau) \\ \vdots \\ \sin(e_b^{N_b}\tau) 
	\end{bmatrix} 
\end{align*}
and the dynamical systems \eqref{eq:Cauchy} can be rewritten in a vectorial form as:
\begin{align}\label{eq:CauchyVec}
	\begin{cases}
		\displaystyle \dot{\bm{\alpha}}(\tau) = \frac 2\pi \bm{\mathcal{D}}^\alpha(\tau) u(\tau), & \tau \in [0,\pi)
		\\[6pt]
		\bm{\alpha}(0) = 0
	\end{cases} \notag
	\\
	\\
	\begin{cases}
		\displaystyle\dot{\bm{\beta}}(\tau)  = \frac 2\pi \bm{\mathcal{D}}^\beta(\tau) u(\tau), & \tau \in [0,\pi) 
		\\[6pt]
		\bm{\beta}(0) = 0
	\end{cases}\notag 
\end{align}
Compressing the notation even more, we can now denote 
\begin{align*}
	\bm{x}(\tau) = \begin{bmatrix} \bm{\alpha}(\tau) \\ \bm{\beta}(\tau) \end{bmatrix}, \quad
	\bm{\mathcal{D}}(\tau) = \begin{bmatrix} \bm{\mathcal{D}}^\alpha(\tau) \\ \bm{\mathcal{D}}^\beta(\tau) \end{bmatrix}     
\end{align*}
so that \eqref{eq:CauchyVec} becomes
\begin{align}\label{eq:CauchyCompact}
	\begin{cases}
		\displaystyle\dot{\bm{x}}(\tau) = \frac 2\pi\bm{\mathcal{D}}(\tau) u(\tau),  & \tau \in [0,\pi)
		\\[6pt]
		\bm{x}(0) = {0}
	\end{cases}
\end{align}
and the target coefficients of the SHE problem will be given by $\bm{x}_T:=[\bm{a}_T,\bm{b}_T]^\top=\bm{x}(\pi)$.

Taking into account this new formulation, as we shall see in more detail in Section \ref{Section4}, Problem \ref{SHEp} can now be recast as a control one for the dynamical systems \eqref{eq:CauchyCompact}, in which we look for a control function $u(\tau)$ steering the state $\bm{x}(\tau)$ from the origin to the target $\bm{x}_T:=[\bm{a}_T,\bm{b}_T]^\top$ in time $\tau = \pi$.

Moreover, since most often control problems are designed to drive the state of a given dynamical system to an equilibrium configuration, for instance the zero state, we introduce the change of variables $\bm{x}(\tau)\mapsto \bm{x}_T - \bm{x}(\tau)$ which allows us to reverse the time in \eqref{eq:CauchyCompact}, thus obtaining 
\begin{equation}\label{eq:CauchyReversed}
    \begin{cases}
        \displaystyle\dot{\bm{x}}(\tau) = -\frac 2\pi\bm{\mathcal{D}}(\tau)u(\tau),  & \tau \in [0,\pi)
        \\[6pt]
        \bm{x}(0) = \bm{x}_T
    \end{cases},
\end{equation}
In this new configuration, the control function $u$ is now required to steer the solution of \eqref{eq:CauchyReversed} from the initial datum $\bm{x}_T$ to zero in time $\tau=\pi$. 

We can then formulate the following control problem, which is equivalent to Problem \ref{SHEp}:
\newline 
\begin{problem}\label{SHEpControl}
Let $\mathcal{U}$ be defined as in \eqref{eq:Udef} and let $\mathcal{E} _a $ and $\mathcal{E} _b $ be two sets of odd numbers with cardinalities $|\mathcal{E}_a| = N_a $ and $ |\mathcal{E} _b| = N_b$, respectively. Given the vectors $\bm{a}_T \in \mathbb{R}^{N_a}$ and $\bm{b}_T \in \mathbb{R}^{N_b} $, let us define $\bm{x}_T=[\bm{a}_T,\bm{b}_T]^\top \in \mathbb{R}^{N_a\times N_b}$. We look for $u:\in [0,\pi)\to\mathcal{U}$ such that the solution of \eqref{eq:CauchyReversed} with initial datum $\bm{x}(0)=\bm{x}_T$ satisfies $\bm{x}(\pi)=0$.
\end{problem}
$\newline$
\begin{remark}[Quarter-wave symmetry]
We shall mention that, in the SHE literature, it is usual to distinguish among the half-wave symmetry problem (addressed in the present paper) and the quarter-wave symmetry one in which
\begin{align*}
	u\left(\tau + \frac \pi2\right) = -u(\tau)\quad \mbox{for all}\; \tau \in \left[0,\frac \pi2\right).
\end{align*}
In quarter-wave symmetry, the SHE problem simplifies, as the Fourier coefficients $\{a_i\}_{i\in\mathcal E_a}$ turn out to be all zero. Hence, only the phases of the converter's signal can be controlled, while the half-wave SHE allows to deal with the amplitudes as well. It is worth to remark nonetheless that our optimal control formulation can be easily adapted to the quarter-wave symmetry problem by simply replacing the Fourier coefficients \eqref{eq:an} with
\begin{align*}
	a_i = 0, \quad\quad b_j = \frac{4}{\pi} \int_0^{\frac \pi4} u(\tau)  \sin(j \tau)\,d\tau.
\end{align*}
\end{remark}
\section{Optimal control for SHE}\label{Section4}

As we anticipated in Section \ref{Section3}, the SHE problem is equivalent to controlling a dynamical system associated with the Fourier coefficients \eqref{eq:an}. In this section, we present a rigorous formulation of this mentioned control problem and we analyze some relevant properties. 

In what follows, for a given vector $\bm{v}\in\mathbb{R}^d$, we shall always denote by $\|\bm{v}\|$ the euclidean norm $\|\bm{v}\|_{\mathbb{R}^d}$.
\newline


\begin{problem}[OCP for SHE]\label{pb:OCP1}
Let $\mathcal U$ be defined as in \eqref{eq:Udef}. Given two sets of odd numbers $\mathcal {E}_a $ and $\mathcal {E}_b $ with cardinality $N_a$ and $N_b$, respectively, and given the target $\bm{x}_T\in \mathbb{R}^{N_a+N_b}$, we look for the function $u(\tau):[0,\pi)\to \mathcal U$ solution of the optimal control problem  
\begin{align*}
	&\min_{u \in \mathcal{U}}\;\frac 12 \|\bm{x}(\pi)\|^2
	\\
    &\notag \text{subject to: }\quad \begin{cases}
            \displaystyle \dot{\bm{x}}(\tau) = -\frac 2\pi\bm{\mathcal{D}}(\tau) u(\tau),  & \tau \in [0,\pi)\\[6pt]
            \bm{x}(0) = \bm{x}_0
    \end{cases}
    \end{align*}
\end{problem}
The solution of Problem \ref{pb:OCP1} may be quite complex to be obtained, due to the restriction on the admissible control values. 

\begin{definition}[Digital control of set $\mathcal{U}$]
A control $u(\tau)$ is called digital if, for each time $\tau\geq 0$, it only takes values in the finite set of real number $\mathcal{U}$.  
\end{definition}
\textcolor{red}{¿Igual el nombre de la definicion debe depender de $\mathcal{U}$?, Es decir, `Digital control of set $\mathcal{U}$'}

\textcolor{red}{Falta caracterizar los instantes de instantes de tiempo donde $u(\tau) \notin \mathcal{U}$. Estos pueden existir pero el conjunto de estos puntos debe ser de medida nula}


In order to bypass this difficulty, following a standard optimal control approach, we can formulate an equivalent minimization problem in which, instead of looking for $u\in\mathcal U$, we simply require that $|u|<1$ and we introduce a penalization term to ensure that $u$ is a piece-wise constant function (\textcolor{red}{digital control}). This alternative optimal control problem, which can be solved more easily by employing standard tools, reads as follows:
\newline
\begin{problem}[Penalized OCP for SHE]\label{pb:OCP2}
Fix $\epsilon>0$. Given two sets of odd numbers $\mathcal E_a $ and $\mathcal E_b $ and the target $\bm{x}_T \in \mathbb {R}^{N_a + N_b}$, we look for the  function $u:[0,\pi)\to\mathcal U$ as the solution of:
\begin{align*}
	&\min_{|u|<1} \Bigg[\Psi(\bm{x}) + \epsilon \int_0^{\pi} \mathcal{L}(u(\tau)) d\tau \Bigg]  
	\\[5pt]
	&\Psi(\bm{x}) = \frac 12 \|\bm{x}(\pi) \|^2,
\end{align*}
under the dynamics given by \eqref{eq:CauchyReversed}.
\end{problem}

In Problem \ref{pb:OCP2}, the penalization function $\mathcal L: \mathbb{R} \rightarrow \mathbb{R}$ will be chosen such that the optimal control $u^*$ only takes values in $\mathcal U$. Furthermore, the parameter $\epsilon$ should be small so that the solution minimizes the distance from the final state and the target.

Next we will study the optimality conditions of the problem, for a general $\mathcal L$, and then specify how $\mathcal L$ should be so that the optimal control $u^*$ only takes the allowed values in $\mathcal U$.

\subsection{Optimality conditions}

To write the optimality conditions of the problem we will use the Pontryagin minimum principle \cite[Chapter~2.7]{bryson1975applied}. With this purpose, it is necessary to first introduce the Hamiltonian function 
\begin{align*}\label{eq:hamil}
    H(u,\bm{p},\tau) = \epsilon \mathcal{L}(u) - \frac 2\pi\big(\bm{p}(\tau) \cdot \bm{\mathcal{D}}(\tau)\big)u(\tau),
\end{align*}
where $\bm{p}(\tau)$ is the so-called adjoint state, which is associated with the restriction imposed by the system. This vector has the same dimension of the state $\bm{x}$, so that
\begin{gather}
  \bm{x}(\tau) = \begin{bmatrix} \bm{\alpha}(\tau) \\ \bm{\beta}(\tau) \end{bmatrix} \Leftrightarrow 
  \bm{p}(\tau) = \begin{bmatrix} \bm{p}^\alpha(\tau) \\ \bm{p}^\beta(\tau) \end{bmatrix}.
\end{gather}
In what follows, we will enumerate the optimality conditions arising from the Pontryagin principle.
\begin{itemize}
    \item[1.] \textbf{Adjoint equation}: the ODE describing the evolution of the adjoint variable is given by 
    \begin{align*}
    	\dot{\bm{p}}(\tau) = -\nabla_x H(u(\tau),\bm{p}(\tau),\tau).
    \end{align*}
    In our case, since the Hamiltonian does not depend on the dynamics, we simply have
    \begin{align}\label{eq:equationP}
    	\dot{\bm{p}}(\tau) = 0,
    \end{align}
	that is, the adjoint state is constant in time.
	
	\item[2.] \textbf{Final condition of the adjoint}: As it is well-known, the adjoint equation is defined backward in time, meaning that its initial condition is actually a final one, posed at $\tau=\pi$. This final condition is given by 
    \begin{align*}
    	\bm{p}(\pi) = \nabla_{\bm{x}} \Psi(\bm{x}) = \bm{x}(\pi) - \bm{x}_T.
    \end{align*} 
	This, together with \eqref{eq:equationP}, tells us that
	\begin{align*}
		\bm{p}(\tau) = \bm{x}(\pi) - \bm{x}_T, \quad \mbox{ for all }\tau\in [0,\pi).
	\end{align*} 
    
    \item[3.] \textbf{Optimal  Waveform}: We known that 
    \begin{align*}
    	u^* = \argmin_{|u|<1} H(\tau,\bm{p}^*,u),
    \end{align*}
	so that, in this case, we can write
    \begin{gather}
        u^*(\tau) = \argmin_{|u|<1}  \left[\epsilon \mathcal{L}(u(\tau)) - \frac 2\pi \big(\bm{p}^* \cdot \bm{\mathcal{D}}(\tau)\big) u(\tau) \right].
    \end{gather}
    Therefore, this optimality condition reduces to the optimization of a function in a variable within the interval $ [- 1,1] $. 
\end{itemize}
\vspace{1em}
\begin{definition}
    Dado el problema \ref{pb:OCP2} definimos una función $\mathcal{H}_m:[-1,1]\rightarrow \mathbb{R}$ tal que:
    \begin{gather}\label{Hm}
        \mathcal{H}_m(u) = \epsilon \mathcal{L}(u) - mu  |  \forall m \in \mathbb{R}
    \end{gather}
\end{definition}
Es importante notar que la función $\mathcal{H}_m$ es el Hamiltoniano del sistema donde hemos remplazado el valor 
\begin{gather}
	[\bm{p}^* \cdot \bm{\mathcal{D}}(\tau)] = \sum_{i \in \mathcal{E}_a} p^*_\alpha \cos(i\tau) + \sum_{j \in \mathcal{E}_b} p^*_\beta \sin(j\tau) 
\end{gather}
por el parámetro $m$. De manera que el Hamiltoniano evaluado en la trayectoria óptima varía de manera continua en todo el intervalo $\tau \in [0,\pi)]$. 
%
Esta es la razón por la que el estudio de la función $\mathcal{H}_m$, una función uni-variable parametrizada por $m$ ,tiene implicaciones en el Problema \ref{pb:OCP2}.
\newline

\begin{definition}
    Dado el Problema \ref{pb:OCP2}  definimos una función $\mathcal{G}:\mathbb{R} \rightarrow [-1,1]$ tal que:
    \begin{gather}
        \mathcal{G}(m) = \argmin_{u \in [-1,1]} \mathcal{H}_m(u)
    \end{gather}
\end{definition}
\begin{definition}
    Dado el Problema \ref{pb:OCP2} definimos el conjunto $\mathcal{M}$ como:
    \begin{gather}
        \mathcal{M} = \{m \in \mathbb{R}\ | \ \mathcal{G}(m) \notin \mathcal{U} \}
    \end{gather}
\end{definition}
\subsection{Caso Binivel (Control Bang-Bang)}
El caso binivel es el que tiene como conjunto admisible de controles $\mathcal{U}= \{-1,1\}$.
\vspace{1em}

\begin{proposition}
    Dado el Problema \ref{pb:OCP2} con el conjunto admisible de control $\mathcal{U} = \{-1,1\}$. Si la función $\mathcal{L}$ es concava en el intervalo $[-1,1]$ del Problema \ref{pb:OCP2}, entonces la solución de problema es un control digital del conjunto $\mathcal{U} =  \{-1,1\}$
\end{proposition}
\begin{proof}
    Si $\mathcal{L}$ es concava en el intervalo entonces $\mathcal{H}_m$ también lo es. De manera que $G(m)$ solo puede tomar los valores $\{-1,1\}$.
\end{proof}
\subsection{Caso Multi-nivel}
Para el caso multi nivel deben existir $m$ para los cuales $H_m$ tenga un mínimo dentro del intervalo $[-1,1]$. Además este mínimo no puede variar de manera continua con un variación de $m$.
\vspace{1em}
\begin{proposition}
    Si $\mathcal{L}$ es derivable entonces la solución del Problema \ref{pb:OCP2} no es un control digital.
\end{proposition}
\begin{proof}
    Dado que $\mathcal{L}$ es derivable en todo el intervalo $[-1,1]$ también lo es $\mathcal{H}_m$. De manera que podemos derivar la función $\mathcal{H}_m$.
    \begin{gather}
        \frac{d \mathcal{H}_m}{du} = 0 \rightarrow
        \frac{d \mathcal{L}}{du} = m
    \end{gather}
\end{proof}
\vspace{1em}
\begin{proposition}
    Si la solución del Problema \ref{pb:OCP2} es un control digital, entonces $\mathcal{M}$ es el conjunto vacio o  un conjunto finito de elementos. 
\end{proposition}
\begin{proof}

En el caso que $\mathcal{M} = \{\emptyset\}$ por definición de $\mathcal{M}$ la imagen de $\mathcal{G}(m)$ es el conjunto $\mathcal{U}$ de manera que la solución del Problema \ref{pb:OCP2} es un control digital de $\mathcal{U}$.  

Fuera del caso anterior, si suponemos que el conjunto $\mathcal{M}$ contiene algún subintervalo $\mathcal{I}_{\mathcal{M}} \subset[-1,1]$ podemos tomar    $m_1 \in \mathcal{I}_{\mathcal{M}}$ y $ m_2 = m_1 + \varepsilon \ / \ \varepsilon << 1$ de modo que se debe cumplir que $m_2 \in \mathcal{I}_{\mathcal{M}}$
\begin{equation}
    \begin{aligned}
        \mathcal{G}(m_1) = & \argmin_{u\in[-1,1]} \ [\mathcal{H}_{m_1}(u)] & \in \mathcal{U}\\
        \mathcal{G}(m_2) = & \argmin_{u\in[-1,1]} \ [\mathcal{H}_{m_1}(u) + \varepsilon u]   & \in \mathcal{U} 
    \end{aligned}      
\end{equation}

¿Eso implica que $\mathcal{U}$ tentría algún subintervalo continuo de $[-1,1]$?
$\hfill\blacksquare$.
\end{proof} 



In what follows, we will show that there are several ways of designing a penalization term $\mathcal L$ giving us digital controls.
\newline
\begin{theorem}
Assume that the finite set $\mathcal{U}$ defined in \eqref{eq:Udef} is composed by elements in ascending order. Let $\mathcal{Y} = \{y_\ell\}_{\ell=1}^L$ be another finite set, with the same cardinality as $\mathcal U$, such that the $L-1$ tuple $d\mathcal{Y} = \{y_\ell - y_{\ell+1}\}_{\ell=1}^{L-1}$ is monotone. Let $\mathcal{L}:\mathbb{R} \rightarrow \mathbb{R}$ be a piece-wise continuous function en los intervalos $\{ [u_l,u_{l+1})\}_{l=1}^L$ de manera que $\{y_l = \mathcal{L}(u_l)\}_{l=1}^L$. Si la funciones definidas entre cada intervalo $\{[u_l,u_{l+1}]\}_{l=1}^L$ son concavas, entonces la penalización $\mathcal{L}$ en el problema \ref{pb:OCP2} da lugar a un control que solo toma valores en $\mathcal{U}$.
\end{theorem}
Para que el considerar que el control óptimo del problema solo toma valores digitales los minimos  \ref{Hm} para todo valor de $m$ de estar contenido en $\mathcal{U}$. A continuación la prueba
\newline

\begin{proof}
    Teniendo en cuenta que la función 
\end{proof}

\subsubsection{Piecewise linear penalization}

In this section, we discuss how to design the penalization term $\mathcal{L}(u)$ so that the optimal control is always contained in $\mathcal{U}$. 

In more detail, we can choose the affine interpolation of a parabola $\mathcal{L}:[-1,1] \rightarrow \mathbb{R}$ as a penalization term. That is
\begin{gather}\label{PLP}
    \mathcal{L}(u) = \begin{cases}
        \big[ (u_{k+1}+u_{k}) (u-u_k) + u_k^2 \big] & \text{if }  u \in [u_k,u_{k+1}[ \\
        1 & \text{if } u = u_{N_u} 
    \end{cases} \\
    \notag \forall k \in \{1,\dots,N_u-1\}
\end{gather}
%
Nevertheless, to compute the minimum of $\mathcal{H}_m(u)$, we shall take into account that this function is not differentiable and the optimality condition then requires to work with the subdifferential $\partial\mathcal{L}(u)$, which given by
\begin{gather}
        \partial\mathcal{L}(u)= \begin{cases}
            \{u_1 + u_2  \}   & \text{if } \ u = u_1 \\
            %%%%%%%%%%%%%%%%%%%%%%%%%%%%%%%%%%%%%%%%%%%%%%%%
            \{u_k + u_{k+1}\}  & \text{if } \ u \in \ ]u_k,u_{k+1}[ \hspace{0.9em} \dagger\\
            %%%%%%%%%%%%%%%%%%%%%%%%%%%%%%%%%%%%%%%%%%%%%%%%
            [u_k+u_{k-1} ,  u_{k+1}+u_k] & \text{if} \ u = u_k \hspace{3.9em} \ddagger \\
            %%%%%%%%%%%%%%%%%%%%%%%%%%%%%%%%%%%%%%%%%%%%%%%%%
            \{u_{N_u} + u_{N_u-1}  \} & \text{if} \ u = u_{N_u} 
       \end{cases} \\
       \notag \dagger \ \forall k \in \{1,\dots,N_u-1\} \hspace{1em}
       \notag \ddagger  \ \forall k \in \{2,\dots,N_u-1\}
\end{gather} 

Hence, we have $\partial H_m = \epsilon\partial \mathcal{L} - m$. This means that, given $m\in \mathbb{R}$, we look for $u \in [-1,1]$ minimizing $\mathcal{H}_m(u)$. It is then necessary to determine whether zero belongs to $\partial \mathcal{H}_m(u)$.

\begin{itemize}
    \item \textbf{Case 1: $m \leq \epsilon(u_1+u_2)$}: since $m$ is less than the  minimum of all subdifferentials, then zero does not belong to any of the intervals we defined. Hence, the minimum is in one of the extrema
    \begin{gather}
        \argmin_{|u| < 1} \mathcal{H}_m(u) = u_1
    \end{gather} 
    \item \textbf{Case 2: $m = \epsilon(u_{k+1}+u_k) $}: taking into account that $\forall k \in \{1,\dots,N_u-1\}$,
    \begin{gather}
        \argmin_{|u| < 1} \mathcal{H}_m(u) = [u_k,u_{k+1}[ 
    \end{gather} 
    \item \textbf{Case 3: $\epsilon(u_k+u_{k-1})<m<\epsilon(u_{k+1}+u_k)$}: taking into account that $\forall k \in \{2,\dots,N_u-1\}$,
    \begin{gather}
        \argmin_{|u| < 1} \mathcal{H}_m(u) = u_k
    \end{gather}
    \item \textbf{Case 4: $m>\epsilon(u_{N_u}+u_{N_u-1})$}:
    \begin{gather}
        \argmin_{|u| < 1} \mathcal{H}_m(u) = u_{N_u}
    \end{gather} 
\end{itemize}

In other words, only when $m = \epsilon(u_{k+1}+u_k)$ the minima of the Hamiltonian belong to an interval. For all the other values of $m\in\mathbb{R}$, these minima are contained in $\mathcal{U}$. So that under a continuous variation of $m$, Case 2 can only occur pointwise. Recalling the optimal control problem $m(\tau) = [\bm{p}(\tau) \cdot \bm{\mathcal{D}}(\tau)]$, we can notice that Case 2 corresponds to the instants $\tau$ of change of value.
\section{Numerical simulations}\label{Section5}

En esta sección mostraremos algunos ejemplo resolviendo el problema de control óptimo mediante el método directo y la herramienta de optimización no lineal bajo constrain: CasADi \cite{Andersson2019}.
%
\subsection{Smooth approximation of piecewise linear penalization}

Con el fin de utilizar softwre de optimización para resolver el problema de control óptimo planteado aproximaremos penalización lineal a trozos mediante con ayuda de función escalón de Heaviside $h:\mathbb{R} \rightarrow \mathbb{R}$ y su aproximación suave $h^\eta:\mathbb{R} \rightarrow \mathbb{R}$ definida de la siguiente manera: 
\begin{gather}
    h(x) = \begin{cases}
        1 & \text{ if } x \geq 0 \\
        0 & \text{ if } x < 0
    \end{cases}    
    \hspace{2em} 
    \begin{cases}
        h^\eta(x) = (1 + \tanh(\eta x))/2   \\
        \eta \rightarrow \infty
    \end{cases}
\end{gather}
Con ayuda de la función $h$ definimos la función ventana suave $\Pi_{a,b}^\eta:\mathbb{R} \rightarrow \mathbb{R}$ como:
\begin{gather}
    \Pi_{[a,b]}^\eta(x) = - 1 + h^\eta(x-a) + h^\eta(-x+b) 
\end{gather}
De manera simplificada: 
\begin{gather}
    \Pi_{[a,b]}^\eta(x) = \frac{\tanh[\eta( x -a)] + \tanh[\eta (b-x)]}{2}
\end{gather}
De esta menera podemos escribir la versión  suave de (\ref{PLP}):
\begin{gather}
    \mathcal{L}^\eta(u) = \sum_{k = 1}^{N_u-1} \big[ (u_{k+1}+u_{k}) (u-u_k) + u_k^2 \big] \Pi^\eta_{[u_k,u_{k+1}]}(u)
\end{gather}
De manera que cuando $\eta \rightarrow \infty$ entonces $\mathcal{L}^\eta \rightarrow \mathcal{L}$

\subsection{Direct method  for  OCP-SHE}

To solve the optimal control problem (\ref{OCP2}), we use a direct method. 
%
If we consider a partition $\mathcal{P} = \{\tau_0,\tau_1,\dots,\tau_{T}\}$ of interval $[0,T]$ , we can represent a function $\{ u(\tau) \ | \ \tau \in [0,T]\}$ as a vector $\bm{u} \in \mathbb{R}^{T}$ where component $u_t = u(\tau_t)$. 
%
Then the optimal control problem (\ref{OCP1}) can be written as optimization problem with variable $\bm{u} \in \mathbb{R}^{T}$. This problem is a nonlinear programming, for this we use CasADi software to solve. 
%
Entonces, dado una partición del intervalo $[0,\pi)$ podemos reformular el problema (\ref{OCP1}) como el siguiente problema en tiempo discreto:
\newline
\begin{problem}[OCP numérico]
    Dados dos conjuntos de números impares $\mathcal{E}_a$ and $\mathcal{E}_b$ con cardinalidades $|\mathcal{E}_a| = N_a$ y  $|\mathcal{E}_b| = N_b$ respectivamente, dados los vectores objetivos $\bm{a}_T  \in \mathbb{R}^{N_a}$, de manera que $\bm{x}_0 = [\bm{a}_T,\bm{b}_T]^T$; y $\bm{b}_T  \in \mathbb{R}^{N_b}$ y una  partition $\mathcal{P}_\tau = \{\tau_0,\tau_1,\dots,\tau_{T}\}$ of interval $[0,\pi)$. We search a vector $\bm{u} \in \mathbb{R}^{T}$ that minimize the following function:
    \begin{gather}
        \min_{\bm{u} \in \mathbb{R}^{T} } 
        \Bigg[ 
        || \bm{x}^{T}||^2
        + \epsilon  \sum_{t=0}^{T-1} \mathcal{L}^\eta(u_{t}) \Delta\tau_t  \Bigg]  \\
        \notag \text{suject to: } \\
        \forall \tau \in \mathcal{P} \begin{cases}
            \bm{x}^{t+1} = \bm{x}^{t} - (2/\pi)\Delta \tau_t \bm{\mathcal{D}}(\tau_t)u_t \\
            \bm{x}^0 = \bm{x}_0
        \end{cases} 
    \end{gather}
\end{problem}



\subsection{Resultados}

Cabe mencionar que todas las simulaciones se han realizado con un ordenador de mesa con $8Gb$ de ram, y el tiempo de ejecución para la búsqueda de soluciones dado un vector objetivo es del orden del segundo. A continuación listaremos cada uno de los resultados numéricos obtenidos:
\begin{enumerate}    


    \item \textbf{OCP con simetría de cuarto de onda}: Consideraremos el problema  con un conjunto de números impares $\mathcal{E}_b = \{1,5\}$ con una discretización del intervalo $[0,\pi/2]$ de $T = 200$. Mostramos las soluciones para los vectores objetivo $b_T^1 = \{(-0.4,-0.3,\dots,0.3,0.4)\}$ manteniendo $b_T^5=0$ para todos los casos. Mostramos las trayectorias óptimas obtenidas en la figura (\ref{ex01}), donde se puede ver una continuidad en las soluciones con respecto a vector objetivo.

    % \begin{figure}
    %     \centering
    %     \begin{subfigure}[b]{\textwidth}
    %         \centering
    %         \includegraphics[width=0.6\textwidth]{img/ex01-con.eps}
    %         \caption{Dynamical System: el punto rojo hace referencia al punto final mientras que el punto negro hace referencia al punto inicial.}
    %     \end{subfigure} 
    %     \hfill \\
    %     \begin{subfigure}[b]{\textwidth}
    %         \centering
    %         \includegraphics[width=0.6\textwidth]{img/ex01-dyn.eps} 
    %         \caption{Control}
    %     \end{subfigure}
    %     \caption{Mostramos las trayectorias óptimas y controles óptimos para distintos vectores objetivo.}
    %     \label{ex01}
    % \end{figure}


    \item \textbf{OCP con simetría de cuarto onda para un intervalo del $b_1$}: Para este ejemplo consideramos el siguiente conjunto de números impares: $\mathcal{E}_b = \{1,5,7,11,13\}$. 
    %
    Ademas consideramos el vector objetivo $\bm{b}_T = [m_a,0,0,0,0]$, donde  $m_a \in [-1,1]$ es un parámetro. with three penalization terms: $\mathcal{L}(u) = -f$, $\mathcal{L}(u) = +f$ and $\mathcal{L}(u) = -f^2$ obtained by direct method with uniform partition of interval $[0,\pi/2]$ with $T=400$ and penalization parameter $\epsilon = 10^{-5}$. 
    %
    Para cada uno de los términos de penalización utilizados la distancia entre los coeficientes de Fourier se encuentra en el orden de $10^{-4}$. 
    %
    Sin embargo, cuando el término de penalización es $\mathcal{L}(u)= -f^2$ la solución no presenta continudad con respecto al vector objetivo. 
    %
    Por otra parte, es importante mencionar que las soluciones para los términos de penalización $\mathcal{L}(u) = -f$ y $\mathcal{L}(u) = f$ cumplen una simetría por lo que invirtiendo las soluciones con respecto al origen y invirtiendo el signo de las soluciones se puede ver que ambas soluciones son la misma.


     
    \item \textbf{SHE para tres niveles}: Podemos ver que en el caso en el que el control $u(\tau)$ solo pueda tomar valores entre $[0,1]$ obtenemos señales que pueden tomar tres niveles en el intervalo $[0,2\pi]$ gracias a la simetría de cuarto de onda. Si resolvemos el problema de control óptimo pero esta vez cambiando las restricciones $|u(\tau)|<1$ por $\{0<u(\tau)<1\}$. Se ha realizado el mismo procedimiento que en el caso anterior, obteniendo soluciones para los mismo términos de penalización obteniendo la figura (\ref{ex3LVL}). Allí se muestra la continudad de las soluciones y que estas se encuentran en el orden de $10^{-4}$.
    



    
      


    \item \textbf{Cambio en el número de conmunationes}: Gracias a la formulación de control óptimo para el problema SHE podemos variar el número de ángulos de conmuntación. 
    %
    Este es el cado del siguientes ejemplo, donde hemos tomado como conjunto de números pares $\mathcal{E}_b = \{1,3,9,13,17\}$,   además consideramos el vector objetivo $\bm{b}_T = [m_a,0,0,0,0]$, donde  $m_a \in [0,1]$ es un parámetro. 
    %
    En este problema hemos utilizado una penalización tipo $\mathcal{L} = f$ con un parémetro de penalización $\epsilon=10^{-4}$.
    %
    Podemos ver en la figura (\ref{disco}) como el problema de control óptimo es versátil y es capaz de mover entre varios conjuntos de soluciones.




    
    \item \textbf{OCP para SHE con simetría de media onda}: Se ha relizado el caso de control óptimo de media onda con con $\mathcal{E}_a = \{1,3,5\}$ y  $\mathcal{E}_b = \{1,3,5,9\}$, donde $\bm{a}_T = [m_a,0,0]$, $\bm{b}_T = [m_a,m_a,0,0]$ y  $m_a \in [-0.6,0.6]$. Se ha elegido la penalización $L(u) = +f$



\end{enumerate}







\section{Conclusiones}\label{Section6}


We presented the SHE problem from the point of view of control theory. This methodology allows obtaining a $10^{-4}-10^{-5}$ precision in the distance to the target vector. Nevertheless, comparing with methodologies where the commutation number is fixed a priori, our approximation is computationally more expensive. Notwithstanding that, the optimal control provides solutions in the entire range of the modulation index, although the number of solutions or their locations change dramatically.

This methodology for the SHE problem connects control theory with harmonic elimination. In this way, the SHE problem can be solved through classical tools.

\begin{ack}                               % Place acknowledgements
Partially supported by the Roman Senate.  % here.
\end{ack}
 
\bibliographystyle{apalike}        % Include this if you use bibtex 
\bibliography{bib}           % and a bib file to produce the 
                                 % bibliography (preferred). The
                                 % correct style is generated by
                                 % Elsevier at the time of printing.




\end{document}